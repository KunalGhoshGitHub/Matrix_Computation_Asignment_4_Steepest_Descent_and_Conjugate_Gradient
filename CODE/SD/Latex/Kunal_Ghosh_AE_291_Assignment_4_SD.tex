\documentclass[11pt]{article}

    \usepackage[breakable]{tcolorbox}
    \usepackage{parskip} % Stop auto-indenting (to mimic markdown behaviour)
    

    % Basic figure setup, for now with no caption control since it's done
    % automatically by Pandoc (which extracts ![](path) syntax from Markdown).
    \usepackage{graphicx}
    % Maintain compatibility with old templates. Remove in nbconvert 6.0
    \let\Oldincludegraphics\includegraphics
    % Ensure that by default, figures have no caption (until we provide a
    % proper Figure object with a Caption API and a way to capture that
    % in the conversion process - todo).
    \usepackage{caption}
    \DeclareCaptionFormat{nocaption}{}
    \captionsetup{format=nocaption,aboveskip=0pt,belowskip=0pt}

    \usepackage{float}
    \floatplacement{figure}{H} % forces figures to be placed at the correct location
    \usepackage{xcolor} % Allow colors to be defined
    \usepackage{enumerate} % Needed for markdown enumerations to work
    \usepackage{geometry} % Used to adjust the document margins
    \usepackage{amsmath} % Equations
    \usepackage{amssymb} % Equations
    \usepackage{textcomp} % defines textquotesingle
    % Hack from http://tex.stackexchange.com/a/47451/13684:
    \AtBeginDocument{%
        \def\PYZsq{\textquotesingle}% Upright quotes in Pygmentized code
    }
    \usepackage{upquote} % Upright quotes for verbatim code
    \usepackage{eurosym} % defines \euro

    \usepackage{iftex}
    \ifPDFTeX
        \usepackage[T1]{fontenc}
        \IfFileExists{alphabeta.sty}{
              \usepackage{alphabeta}
          }{
              \usepackage[mathletters]{ucs}
              \usepackage[utf8x]{inputenc}
          }
    \else
        \usepackage{fontspec}
        \usepackage{unicode-math}
    \fi

    \usepackage{fancyvrb} % verbatim replacement that allows latex
    \usepackage{grffile} % extends the file name processing of package graphics
                         % to support a larger range
    \makeatletter % fix for old versions of grffile with XeLaTeX
    \@ifpackagelater{grffile}{2019/11/01}
    {
      % Do nothing on new versions
    }
    {
      \def\Gread@@xetex#1{%
        \IfFileExists{"\Gin@base".bb}%
        {\Gread@eps{\Gin@base.bb}}%
        {\Gread@@xetex@aux#1}%
      }
    }
    \makeatother
    \usepackage[Export]{adjustbox} % Used to constrain images to a maximum size
    \adjustboxset{max size={0.9\linewidth}{0.9\paperheight}}

    % The hyperref package gives us a pdf with properly built
    % internal navigation ('pdf bookmarks' for the table of contents,
    % internal cross-reference links, web links for URLs, etc.)
    \usepackage{hyperref}
    % The default LaTeX title has an obnoxious amount of whitespace. By default,
    % titling removes some of it. It also provides customization options.
    \usepackage{titling}
    \usepackage{longtable} % longtable support required by pandoc >1.10
    \usepackage{booktabs}  % table support for pandoc > 1.12.2
    \usepackage{array}     % table support for pandoc >= 2.11.3
    \usepackage{calc}      % table minipage width calculation for pandoc >= 2.11.1
    \usepackage[inline]{enumitem} % IRkernel/repr support (it uses the enumerate* environment)
    \usepackage[normalem]{ulem} % ulem is needed to support strikethroughs (\sout)
                                % normalem makes italics be italics, not underlines
    \usepackage{mathrsfs}
    

    
    % Colors for the hyperref package
    \definecolor{urlcolor}{rgb}{0,.145,.698}
    \definecolor{linkcolor}{rgb}{.71,0.21,0.01}
    \definecolor{citecolor}{rgb}{.12,.54,.11}

    % ANSI colors
    \definecolor{ansi-black}{HTML}{3E424D}
    \definecolor{ansi-black-intense}{HTML}{282C36}
    \definecolor{ansi-red}{HTML}{E75C58}
    \definecolor{ansi-red-intense}{HTML}{B22B31}
    \definecolor{ansi-green}{HTML}{00A250}
    \definecolor{ansi-green-intense}{HTML}{007427}
    \definecolor{ansi-yellow}{HTML}{DDB62B}
    \definecolor{ansi-yellow-intense}{HTML}{B27D12}
    \definecolor{ansi-blue}{HTML}{208FFB}
    \definecolor{ansi-blue-intense}{HTML}{0065CA}
    \definecolor{ansi-magenta}{HTML}{D160C4}
    \definecolor{ansi-magenta-intense}{HTML}{A03196}
    \definecolor{ansi-cyan}{HTML}{60C6C8}
    \definecolor{ansi-cyan-intense}{HTML}{258F8F}
    \definecolor{ansi-white}{HTML}{C5C1B4}
    \definecolor{ansi-white-intense}{HTML}{A1A6B2}
    \definecolor{ansi-default-inverse-fg}{HTML}{FFFFFF}
    \definecolor{ansi-default-inverse-bg}{HTML}{000000}

    % common color for the border for error outputs.
    \definecolor{outerrorbackground}{HTML}{FFDFDF}

    % commands and environments needed by pandoc snippets
    % extracted from the output of `pandoc -s`
    \providecommand{\tightlist}{%
      \setlength{\itemsep}{0pt}\setlength{\parskip}{0pt}}
    \DefineVerbatimEnvironment{Highlighting}{Verbatim}{commandchars=\\\{\}}
    % Add ',fontsize=\small' for more characters per line
    \newenvironment{Shaded}{}{}
    \newcommand{\KeywordTok}[1]{\textcolor[rgb]{0.00,0.44,0.13}{\textbf{{#1}}}}
    \newcommand{\DataTypeTok}[1]{\textcolor[rgb]{0.56,0.13,0.00}{{#1}}}
    \newcommand{\DecValTok}[1]{\textcolor[rgb]{0.25,0.63,0.44}{{#1}}}
    \newcommand{\BaseNTok}[1]{\textcolor[rgb]{0.25,0.63,0.44}{{#1}}}
    \newcommand{\FloatTok}[1]{\textcolor[rgb]{0.25,0.63,0.44}{{#1}}}
    \newcommand{\CharTok}[1]{\textcolor[rgb]{0.25,0.44,0.63}{{#1}}}
    \newcommand{\StringTok}[1]{\textcolor[rgb]{0.25,0.44,0.63}{{#1}}}
    \newcommand{\CommentTok}[1]{\textcolor[rgb]{0.38,0.63,0.69}{\textit{{#1}}}}
    \newcommand{\OtherTok}[1]{\textcolor[rgb]{0.00,0.44,0.13}{{#1}}}
    \newcommand{\AlertTok}[1]{\textcolor[rgb]{1.00,0.00,0.00}{\textbf{{#1}}}}
    \newcommand{\FunctionTok}[1]{\textcolor[rgb]{0.02,0.16,0.49}{{#1}}}
    \newcommand{\RegionMarkerTok}[1]{{#1}}
    \newcommand{\ErrorTok}[1]{\textcolor[rgb]{1.00,0.00,0.00}{\textbf{{#1}}}}
    \newcommand{\NormalTok}[1]{{#1}}

    % Additional commands for more recent versions of Pandoc
    \newcommand{\ConstantTok}[1]{\textcolor[rgb]{0.53,0.00,0.00}{{#1}}}
    \newcommand{\SpecialCharTok}[1]{\textcolor[rgb]{0.25,0.44,0.63}{{#1}}}
    \newcommand{\VerbatimStringTok}[1]{\textcolor[rgb]{0.25,0.44,0.63}{{#1}}}
    \newcommand{\SpecialStringTok}[1]{\textcolor[rgb]{0.73,0.40,0.53}{{#1}}}
    \newcommand{\ImportTok}[1]{{#1}}
    \newcommand{\DocumentationTok}[1]{\textcolor[rgb]{0.73,0.13,0.13}{\textit{{#1}}}}
    \newcommand{\AnnotationTok}[1]{\textcolor[rgb]{0.38,0.63,0.69}{\textbf{\textit{{#1}}}}}
    \newcommand{\CommentVarTok}[1]{\textcolor[rgb]{0.38,0.63,0.69}{\textbf{\textit{{#1}}}}}
    \newcommand{\VariableTok}[1]{\textcolor[rgb]{0.10,0.09,0.49}{{#1}}}
    \newcommand{\ControlFlowTok}[1]{\textcolor[rgb]{0.00,0.44,0.13}{\textbf{{#1}}}}
    \newcommand{\OperatorTok}[1]{\textcolor[rgb]{0.40,0.40,0.40}{{#1}}}
    \newcommand{\BuiltInTok}[1]{{#1}}
    \newcommand{\ExtensionTok}[1]{{#1}}
    \newcommand{\PreprocessorTok}[1]{\textcolor[rgb]{0.74,0.48,0.00}{{#1}}}
    \newcommand{\AttributeTok}[1]{\textcolor[rgb]{0.49,0.56,0.16}{{#1}}}
    \newcommand{\InformationTok}[1]{\textcolor[rgb]{0.38,0.63,0.69}{\textbf{\textit{{#1}}}}}
    \newcommand{\WarningTok}[1]{\textcolor[rgb]{0.38,0.63,0.69}{\textbf{\textit{{#1}}}}}


    % Define a nice break command that doesn't care if a line doesn't already
    % exist.
    \def\br{\hspace*{\fill} \\* }
    % Math Jax compatibility definitions
    \def\gt{>}
    \def\lt{<}
    \let\Oldtex\TeX
    \let\Oldlatex\LaTeX
    \renewcommand{\TeX}{\textrm{\Oldtex}}
    \renewcommand{\LaTeX}{\textrm{\Oldlatex}}
    % Document parameters
    % Document title
    \title{Kunal\_Ghosh\_AE\_291\_Assignment\_4\_SD}
    
    
    
    
    
% Pygments definitions
\makeatletter
\def\PY@reset{\let\PY@it=\relax \let\PY@bf=\relax%
    \let\PY@ul=\relax \let\PY@tc=\relax%
    \let\PY@bc=\relax \let\PY@ff=\relax}
\def\PY@tok#1{\csname PY@tok@#1\endcsname}
\def\PY@toks#1+{\ifx\relax#1\empty\else%
    \PY@tok{#1}\expandafter\PY@toks\fi}
\def\PY@do#1{\PY@bc{\PY@tc{\PY@ul{%
    \PY@it{\PY@bf{\PY@ff{#1}}}}}}}
\def\PY#1#2{\PY@reset\PY@toks#1+\relax+\PY@do{#2}}

\@namedef{PY@tok@w}{\def\PY@tc##1{\textcolor[rgb]{0.73,0.73,0.73}{##1}}}
\@namedef{PY@tok@c}{\let\PY@it=\textit\def\PY@tc##1{\textcolor[rgb]{0.24,0.48,0.48}{##1}}}
\@namedef{PY@tok@cp}{\def\PY@tc##1{\textcolor[rgb]{0.61,0.40,0.00}{##1}}}
\@namedef{PY@tok@k}{\let\PY@bf=\textbf\def\PY@tc##1{\textcolor[rgb]{0.00,0.50,0.00}{##1}}}
\@namedef{PY@tok@kp}{\def\PY@tc##1{\textcolor[rgb]{0.00,0.50,0.00}{##1}}}
\@namedef{PY@tok@kt}{\def\PY@tc##1{\textcolor[rgb]{0.69,0.00,0.25}{##1}}}
\@namedef{PY@tok@o}{\def\PY@tc##1{\textcolor[rgb]{0.40,0.40,0.40}{##1}}}
\@namedef{PY@tok@ow}{\let\PY@bf=\textbf\def\PY@tc##1{\textcolor[rgb]{0.67,0.13,1.00}{##1}}}
\@namedef{PY@tok@nb}{\def\PY@tc##1{\textcolor[rgb]{0.00,0.50,0.00}{##1}}}
\@namedef{PY@tok@nf}{\def\PY@tc##1{\textcolor[rgb]{0.00,0.00,1.00}{##1}}}
\@namedef{PY@tok@nc}{\let\PY@bf=\textbf\def\PY@tc##1{\textcolor[rgb]{0.00,0.00,1.00}{##1}}}
\@namedef{PY@tok@nn}{\let\PY@bf=\textbf\def\PY@tc##1{\textcolor[rgb]{0.00,0.00,1.00}{##1}}}
\@namedef{PY@tok@ne}{\let\PY@bf=\textbf\def\PY@tc##1{\textcolor[rgb]{0.80,0.25,0.22}{##1}}}
\@namedef{PY@tok@nv}{\def\PY@tc##1{\textcolor[rgb]{0.10,0.09,0.49}{##1}}}
\@namedef{PY@tok@no}{\def\PY@tc##1{\textcolor[rgb]{0.53,0.00,0.00}{##1}}}
\@namedef{PY@tok@nl}{\def\PY@tc##1{\textcolor[rgb]{0.46,0.46,0.00}{##1}}}
\@namedef{PY@tok@ni}{\let\PY@bf=\textbf\def\PY@tc##1{\textcolor[rgb]{0.44,0.44,0.44}{##1}}}
\@namedef{PY@tok@na}{\def\PY@tc##1{\textcolor[rgb]{0.41,0.47,0.13}{##1}}}
\@namedef{PY@tok@nt}{\let\PY@bf=\textbf\def\PY@tc##1{\textcolor[rgb]{0.00,0.50,0.00}{##1}}}
\@namedef{PY@tok@nd}{\def\PY@tc##1{\textcolor[rgb]{0.67,0.13,1.00}{##1}}}
\@namedef{PY@tok@s}{\def\PY@tc##1{\textcolor[rgb]{0.73,0.13,0.13}{##1}}}
\@namedef{PY@tok@sd}{\let\PY@it=\textit\def\PY@tc##1{\textcolor[rgb]{0.73,0.13,0.13}{##1}}}
\@namedef{PY@tok@si}{\let\PY@bf=\textbf\def\PY@tc##1{\textcolor[rgb]{0.64,0.35,0.47}{##1}}}
\@namedef{PY@tok@se}{\let\PY@bf=\textbf\def\PY@tc##1{\textcolor[rgb]{0.67,0.36,0.12}{##1}}}
\@namedef{PY@tok@sr}{\def\PY@tc##1{\textcolor[rgb]{0.64,0.35,0.47}{##1}}}
\@namedef{PY@tok@ss}{\def\PY@tc##1{\textcolor[rgb]{0.10,0.09,0.49}{##1}}}
\@namedef{PY@tok@sx}{\def\PY@tc##1{\textcolor[rgb]{0.00,0.50,0.00}{##1}}}
\@namedef{PY@tok@m}{\def\PY@tc##1{\textcolor[rgb]{0.40,0.40,0.40}{##1}}}
\@namedef{PY@tok@gh}{\let\PY@bf=\textbf\def\PY@tc##1{\textcolor[rgb]{0.00,0.00,0.50}{##1}}}
\@namedef{PY@tok@gu}{\let\PY@bf=\textbf\def\PY@tc##1{\textcolor[rgb]{0.50,0.00,0.50}{##1}}}
\@namedef{PY@tok@gd}{\def\PY@tc##1{\textcolor[rgb]{0.63,0.00,0.00}{##1}}}
\@namedef{PY@tok@gi}{\def\PY@tc##1{\textcolor[rgb]{0.00,0.52,0.00}{##1}}}
\@namedef{PY@tok@gr}{\def\PY@tc##1{\textcolor[rgb]{0.89,0.00,0.00}{##1}}}
\@namedef{PY@tok@ge}{\let\PY@it=\textit}
\@namedef{PY@tok@gs}{\let\PY@bf=\textbf}
\@namedef{PY@tok@gp}{\let\PY@bf=\textbf\def\PY@tc##1{\textcolor[rgb]{0.00,0.00,0.50}{##1}}}
\@namedef{PY@tok@go}{\def\PY@tc##1{\textcolor[rgb]{0.44,0.44,0.44}{##1}}}
\@namedef{PY@tok@gt}{\def\PY@tc##1{\textcolor[rgb]{0.00,0.27,0.87}{##1}}}
\@namedef{PY@tok@err}{\def\PY@bc##1{{\setlength{\fboxsep}{\string -\fboxrule}\fcolorbox[rgb]{1.00,0.00,0.00}{1,1,1}{\strut ##1}}}}
\@namedef{PY@tok@kc}{\let\PY@bf=\textbf\def\PY@tc##1{\textcolor[rgb]{0.00,0.50,0.00}{##1}}}
\@namedef{PY@tok@kd}{\let\PY@bf=\textbf\def\PY@tc##1{\textcolor[rgb]{0.00,0.50,0.00}{##1}}}
\@namedef{PY@tok@kn}{\let\PY@bf=\textbf\def\PY@tc##1{\textcolor[rgb]{0.00,0.50,0.00}{##1}}}
\@namedef{PY@tok@kr}{\let\PY@bf=\textbf\def\PY@tc##1{\textcolor[rgb]{0.00,0.50,0.00}{##1}}}
\@namedef{PY@tok@bp}{\def\PY@tc##1{\textcolor[rgb]{0.00,0.50,0.00}{##1}}}
\@namedef{PY@tok@fm}{\def\PY@tc##1{\textcolor[rgb]{0.00,0.00,1.00}{##1}}}
\@namedef{PY@tok@vc}{\def\PY@tc##1{\textcolor[rgb]{0.10,0.09,0.49}{##1}}}
\@namedef{PY@tok@vg}{\def\PY@tc##1{\textcolor[rgb]{0.10,0.09,0.49}{##1}}}
\@namedef{PY@tok@vi}{\def\PY@tc##1{\textcolor[rgb]{0.10,0.09,0.49}{##1}}}
\@namedef{PY@tok@vm}{\def\PY@tc##1{\textcolor[rgb]{0.10,0.09,0.49}{##1}}}
\@namedef{PY@tok@sa}{\def\PY@tc##1{\textcolor[rgb]{0.73,0.13,0.13}{##1}}}
\@namedef{PY@tok@sb}{\def\PY@tc##1{\textcolor[rgb]{0.73,0.13,0.13}{##1}}}
\@namedef{PY@tok@sc}{\def\PY@tc##1{\textcolor[rgb]{0.73,0.13,0.13}{##1}}}
\@namedef{PY@tok@dl}{\def\PY@tc##1{\textcolor[rgb]{0.73,0.13,0.13}{##1}}}
\@namedef{PY@tok@s2}{\def\PY@tc##1{\textcolor[rgb]{0.73,0.13,0.13}{##1}}}
\@namedef{PY@tok@sh}{\def\PY@tc##1{\textcolor[rgb]{0.73,0.13,0.13}{##1}}}
\@namedef{PY@tok@s1}{\def\PY@tc##1{\textcolor[rgb]{0.73,0.13,0.13}{##1}}}
\@namedef{PY@tok@mb}{\def\PY@tc##1{\textcolor[rgb]{0.40,0.40,0.40}{##1}}}
\@namedef{PY@tok@mf}{\def\PY@tc##1{\textcolor[rgb]{0.40,0.40,0.40}{##1}}}
\@namedef{PY@tok@mh}{\def\PY@tc##1{\textcolor[rgb]{0.40,0.40,0.40}{##1}}}
\@namedef{PY@tok@mi}{\def\PY@tc##1{\textcolor[rgb]{0.40,0.40,0.40}{##1}}}
\@namedef{PY@tok@il}{\def\PY@tc##1{\textcolor[rgb]{0.40,0.40,0.40}{##1}}}
\@namedef{PY@tok@mo}{\def\PY@tc##1{\textcolor[rgb]{0.40,0.40,0.40}{##1}}}
\@namedef{PY@tok@ch}{\let\PY@it=\textit\def\PY@tc##1{\textcolor[rgb]{0.24,0.48,0.48}{##1}}}
\@namedef{PY@tok@cm}{\let\PY@it=\textit\def\PY@tc##1{\textcolor[rgb]{0.24,0.48,0.48}{##1}}}
\@namedef{PY@tok@cpf}{\let\PY@it=\textit\def\PY@tc##1{\textcolor[rgb]{0.24,0.48,0.48}{##1}}}
\@namedef{PY@tok@c1}{\let\PY@it=\textit\def\PY@tc##1{\textcolor[rgb]{0.24,0.48,0.48}{##1}}}
\@namedef{PY@tok@cs}{\let\PY@it=\textit\def\PY@tc##1{\textcolor[rgb]{0.24,0.48,0.48}{##1}}}

\def\PYZbs{\char`\\}
\def\PYZus{\char`\_}
\def\PYZob{\char`\{}
\def\PYZcb{\char`\}}
\def\PYZca{\char`\^}
\def\PYZam{\char`\&}
\def\PYZlt{\char`\<}
\def\PYZgt{\char`\>}
\def\PYZsh{\char`\#}
\def\PYZpc{\char`\%}
\def\PYZdl{\char`\$}
\def\PYZhy{\char`\-}
\def\PYZsq{\char`\'}
\def\PYZdq{\char`\"}
\def\PYZti{\char`\~}
% for compatibility with earlier versions
\def\PYZat{@}
\def\PYZlb{[}
\def\PYZrb{]}
\makeatother


    % For linebreaks inside Verbatim environment from package fancyvrb.
    \makeatletter
        \newbox\Wrappedcontinuationbox
        \newbox\Wrappedvisiblespacebox
        \newcommand*\Wrappedvisiblespace {\textcolor{red}{\textvisiblespace}}
        \newcommand*\Wrappedcontinuationsymbol {\textcolor{red}{\llap{\tiny$\m@th\hookrightarrow$}}}
        \newcommand*\Wrappedcontinuationindent {3ex }
        \newcommand*\Wrappedafterbreak {\kern\Wrappedcontinuationindent\copy\Wrappedcontinuationbox}
        % Take advantage of the already applied Pygments mark-up to insert
        % potential linebreaks for TeX processing.
        %        {, <, #, %, $, ' and ": go to next line.
        %        _, }, ^, &, >, - and ~: stay at end of broken line.
        % Use of \textquotesingle for straight quote.
        \newcommand*\Wrappedbreaksatspecials {%
            \def\PYGZus{\discretionary{\char`\_}{\Wrappedafterbreak}{\char`\_}}%
            \def\PYGZob{\discretionary{}{\Wrappedafterbreak\char`\{}{\char`\{}}%
            \def\PYGZcb{\discretionary{\char`\}}{\Wrappedafterbreak}{\char`\}}}%
            \def\PYGZca{\discretionary{\char`\^}{\Wrappedafterbreak}{\char`\^}}%
            \def\PYGZam{\discretionary{\char`\&}{\Wrappedafterbreak}{\char`\&}}%
            \def\PYGZlt{\discretionary{}{\Wrappedafterbreak\char`\<}{\char`\<}}%
            \def\PYGZgt{\discretionary{\char`\>}{\Wrappedafterbreak}{\char`\>}}%
            \def\PYGZsh{\discretionary{}{\Wrappedafterbreak\char`\#}{\char`\#}}%
            \def\PYGZpc{\discretionary{}{\Wrappedafterbreak\char`\%}{\char`\%}}%
            \def\PYGZdl{\discretionary{}{\Wrappedafterbreak\char`\$}{\char`\$}}%
            \def\PYGZhy{\discretionary{\char`\-}{\Wrappedafterbreak}{\char`\-}}%
            \def\PYGZsq{\discretionary{}{\Wrappedafterbreak\textquotesingle}{\textquotesingle}}%
            \def\PYGZdq{\discretionary{}{\Wrappedafterbreak\char`\"}{\char`\"}}%
            \def\PYGZti{\discretionary{\char`\~}{\Wrappedafterbreak}{\char`\~}}%
        }
        % Some characters . , ; ? ! / are not pygmentized.
        % This macro makes them "active" and they will insert potential linebreaks
        \newcommand*\Wrappedbreaksatpunct {%
            \lccode`\~`\.\lowercase{\def~}{\discretionary{\hbox{\char`\.}}{\Wrappedafterbreak}{\hbox{\char`\.}}}%
            \lccode`\~`\,\lowercase{\def~}{\discretionary{\hbox{\char`\,}}{\Wrappedafterbreak}{\hbox{\char`\,}}}%
            \lccode`\~`\;\lowercase{\def~}{\discretionary{\hbox{\char`\;}}{\Wrappedafterbreak}{\hbox{\char`\;}}}%
            \lccode`\~`\:\lowercase{\def~}{\discretionary{\hbox{\char`\:}}{\Wrappedafterbreak}{\hbox{\char`\:}}}%
            \lccode`\~`\?\lowercase{\def~}{\discretionary{\hbox{\char`\?}}{\Wrappedafterbreak}{\hbox{\char`\?}}}%
            \lccode`\~`\!\lowercase{\def~}{\discretionary{\hbox{\char`\!}}{\Wrappedafterbreak}{\hbox{\char`\!}}}%
            \lccode`\~`\/\lowercase{\def~}{\discretionary{\hbox{\char`\/}}{\Wrappedafterbreak}{\hbox{\char`\/}}}%
            \catcode`\.\active
            \catcode`\,\active
            \catcode`\;\active
            \catcode`\:\active
            \catcode`\?\active
            \catcode`\!\active
            \catcode`\/\active
            \lccode`\~`\~
        }
    \makeatother

    \let\OriginalVerbatim=\Verbatim
    \makeatletter
    \renewcommand{\Verbatim}[1][1]{%
        %\parskip\z@skip
        \sbox\Wrappedcontinuationbox {\Wrappedcontinuationsymbol}%
        \sbox\Wrappedvisiblespacebox {\FV@SetupFont\Wrappedvisiblespace}%
        \def\FancyVerbFormatLine ##1{\hsize\linewidth
            \vtop{\raggedright\hyphenpenalty\z@\exhyphenpenalty\z@
                \doublehyphendemerits\z@\finalhyphendemerits\z@
                \strut ##1\strut}%
        }%
        % If the linebreak is at a space, the latter will be displayed as visible
        % space at end of first line, and a continuation symbol starts next line.
        % Stretch/shrink are however usually zero for typewriter font.
        \def\FV@Space {%
            \nobreak\hskip\z@ plus\fontdimen3\font minus\fontdimen4\font
            \discretionary{\copy\Wrappedvisiblespacebox}{\Wrappedafterbreak}
            {\kern\fontdimen2\font}%
        }%

        % Allow breaks at special characters using \PYG... macros.
        \Wrappedbreaksatspecials
        % Breaks at punctuation characters . , ; ? ! and / need catcode=\active
        \OriginalVerbatim[#1,codes*=\Wrappedbreaksatpunct]%
    }
    \makeatother

    % Exact colors from NB
    \definecolor{incolor}{HTML}{303F9F}
    \definecolor{outcolor}{HTML}{D84315}
    \definecolor{cellborder}{HTML}{CFCFCF}
    \definecolor{cellbackground}{HTML}{F7F7F7}

    % prompt
    \makeatletter
    \newcommand{\boxspacing}{\kern\kvtcb@left@rule\kern\kvtcb@boxsep}
    \makeatother
    \newcommand{\prompt}[4]{
        {\ttfamily\llap{{\color{#2}[#3]:\hspace{3pt}#4}}\vspace{-\baselineskip}}
    }
    

    
    % Prevent overflowing lines due to hard-to-break entities
    \sloppy
    % Setup hyperref package
    \hypersetup{
      breaklinks=true,  % so long urls are correctly broken across lines
      colorlinks=true,
      urlcolor=urlcolor,
      linkcolor=linkcolor,
      citecolor=citecolor,
      }
    % Slightly bigger margins than the latex defaults
    
    \geometry{verbose,tmargin=1in,bmargin=1in,lmargin=1in,rmargin=1in}
    
    

\begin{document}
    
    \setcounter{secnumdepth}{0}
    
    

    
    \hypertarget{name-kunal-ghosh}{%
\section{Name: Kunal Ghosh}\label{name-kunal-ghosh}}

    \hypertarget{course-m.tech-aerospace-engineering}{%
\section{Course: M.Tech (Aerospace
Engineering)}\label{course-m.tech-aerospace-engineering}}

    \hypertarget{subject-ae-291-matrix-computations}{%
\section{Subject: AE 291 (Matrix
Computations)}\label{subject-ae-291-matrix-computations}}

    \hypertarget{sap-no.-6000007645}{%
\section{SAP No.: 6000007645}\label{sap-no.-6000007645}}

    \hypertarget{s.r.-no.-05-01-00-10-42-22-1-21061}{%
\section{S.R. No.:
05-01-00-10-42-22-1-21061}\label{s.r.-no.-05-01-00-10-42-22-1-21061}}

    \begin{center}\rule{0.5\linewidth}{0.5pt}\end{center}

    \hypertarget{importing-the-necessary-libraries}{%
\section{Importing the necessary
libraries}\label{importing-the-necessary-libraries}}

    \begin{tcolorbox}[breakable, size=fbox, boxrule=1pt, pad at break*=1mm,colback=cellbackground, colframe=cellborder]
\prompt{In}{incolor}{1}{\boxspacing}
\begin{Verbatim}[commandchars=\\\{\}]
\PY{k+kn}{import} \PY{n+nn}{numpy} \PY{k}{as} \PY{n+nn}{np}
\end{Verbatim}
\end{tcolorbox}

    \begin{tcolorbox}[breakable, size=fbox, boxrule=1pt, pad at break*=1mm,colback=cellbackground, colframe=cellborder]
\prompt{In}{incolor}{2}{\boxspacing}
\begin{Verbatim}[commandchars=\\\{\}]
\PY{k+kn}{from} \PY{n+nn}{matplotlib} \PY{k+kn}{import} \PY{n}{cm}
\end{Verbatim}
\end{tcolorbox}

    \begin{tcolorbox}[breakable, size=fbox, boxrule=1pt, pad at break*=1mm,colback=cellbackground, colframe=cellborder]
\prompt{In}{incolor}{3}{\boxspacing}
\begin{Verbatim}[commandchars=\\\{\}]
\PY{k+kn}{import} \PY{n+nn}{matplotlib}\PY{n+nn}{.}\PY{n+nn}{pyplot} \PY{k}{as} \PY{n+nn}{plt}
\end{Verbatim}
\end{tcolorbox}

    \begin{tcolorbox}[breakable, size=fbox, boxrule=1pt, pad at break*=1mm,colback=cellbackground, colframe=cellborder]
\prompt{In}{incolor}{4}{\boxspacing}
\begin{Verbatim}[commandchars=\\\{\}]
\PY{k+kn}{import} \PY{n+nn}{sys}
\end{Verbatim}
\end{tcolorbox}

    \hypertarget{problem}{%
\section{Problem:}\label{problem}}

\hypertarget{solving-2d-poissons-problem-using-steepest-descent-and-conjugate-gradient-iterative-methods}{%
\subsection{Solving 2D Poisson's problem using Steepest Descent and
Conjugate Gradient iterative
methods}\label{solving-2d-poissons-problem-using-steepest-descent-and-conjugate-gradient-iterative-methods}}

\hypertarget{consider-the-2d-poissons-equation-in-the-domain-omega-01-times-0-1-the-unit-square}{%
\subsection{\texorpdfstring{Consider the 2D Poisson's equation in the
domain \(\Omega\) = {[}0,1{]} \(\times\) {[}0, 1{]}, the unit
square:}{Consider the 2D Poisson's equation in the domain \textbackslash{}Omega = {[}0,1{]} \textbackslash{}times {[}0, 1{]}, the unit square:}}\label{consider-the-2d-poissons-equation-in-the-domain-omega-01-times-0-1-the-unit-square}}

    \[-\frac{\partial^2 u}{\partial x^2 } -\frac{\partial^2 u}{\partial y^2 } = f\qquad in \qquad \Omega, \tag{1}\]

    \hypertarget{with-the-boundary-condition}{%
\subsection{with the boundary
condition}\label{with-the-boundary-condition}}

\[u = g \qquad on \qquad \partial \Omega \tag{2}\]

    \hypertarget{where-f-and-g-are-given-functions-and-ux3c9-represents-the-boundary-of-ux3c9.-eq.-1-can-be-discretized-using-the-centered-finite-difference-method-as-explained-in-the-class.}{%
\subsection{where f and g are given functions, and ∂Ω represents the
boundary of Ω. Eq. 1 can be discretized using the centered Finite
difference method (as explained in the
class).}\label{where-f-and-g-are-given-functions-and-ux3c9-represents-the-boundary-of-ux3c9.-eq.-1-can-be-discretized-using-the-centered-finite-difference-method-as-explained-in-the-class.}}

    \hypertarget{consider-the-case-where-f-0-and-g-is-given-as}{%
\subsection{Consider the case where f = 0, and g is given
as,}\label{consider-the-case-where-f-0-and-g-is-given-as}}

    \[g(x,y) = \begin{cases}
0 \qquad\qquad\qquad if\:x = 0
\\
y \qquad\qquad\qquad if\:x = 1
\\
(x-1)sin(x) \:\:\:\:\:\:\: if\:y = 0
\\
x(2-x) \qquad\:\:\:\:\:\: if\:y = 1
\end{cases}\]

    \hypertarget{write-two-separate-codes-one-for-steepest-descent-sd-and-other-for-conjugate-gradient-cg-for-solving-the-discretized-poissons-equation.-for-both-the-methods-take-the-initial-guess-as-u0-0-and-consider-three-mesh-intervals-h-110-h-120-and-h-140.-the-iterations-should-be-continued-until-the-relative-change-in-the-solution-u-from-one-iteration-to-another-is-less-than-10-8.-more-precisely-stop-the-iterations-when}{%
\subsection{\texorpdfstring{Write two separate codes; one for Steepest
Descent (SD) and other for Conjugate Gradient (CG) for solving the
discretized Poisson's equation. For both the methods, take the initial
guess as \(u^{(0)}\) = 0, and consider three mesh intervals: h = 1/10, h
= 1/20 and h = 1/40. The iterations should be continued until the
relative change in the solution u from one iteration to another is less
than \(10^{-8}\). More precisely, stop the iterations
when}{Write two separate codes; one for Steepest Descent (SD) and other for Conjugate Gradient (CG) for solving the discretized Poisson's equation. For both the methods, take the initial guess as u\^{}\{(0)\} = 0, and consider three mesh intervals: h = 1/10, h = 1/20 and h = 1/40. The iterations should be continued until the relative change in the solution u from one iteration to another is less than 10\^{}\{-8\}. More precisely, stop the iterations when}}\label{write-two-separate-codes-one-for-steepest-descent-sd-and-other-for-conjugate-gradient-cg-for-solving-the-discretized-poissons-equation.-for-both-the-methods-take-the-initial-guess-as-u0-0-and-consider-three-mesh-intervals-h-110-h-120-and-h-140.-the-iterations-should-be-continued-until-the-relative-change-in-the-solution-u-from-one-iteration-to-another-is-less-than-10-8.-more-precisely-stop-the-iterations-when}}

    \[\frac{||u^{(k+1)}-u^{(k)}||_2}{||u^{(k+1)}||_2} < 10^{-8} \tag{3}\]

    \hypertarget{along-with-the-codes-submit-a-short-report-with-the-following-items}{%
\subsection{Along with the codes, submit a short report with the
following
items:}\label{along-with-the-codes-submit-a-short-report-with-the-following-items}}

    \hypertarget{for-each-method-sd-and-cg-for-each-h-plot-the-relative-change-in-the-solution-lhs-of-eq.-3-versus-the-iteration-index-k.-in-the-plot-the-relative-change-in-the-solution-should-be-in-base-10-logarithmic-scale-for-example-see-the-command-semilogy-in-matlab}{%
\subsection{1. For each method (SD and CG), for each ''h'', plot the
relative change in the solution (LHS of Eq. 3) versus the iteration
index (k). In the plot, the relative change in the solution should be in
base-10 logarithmic scale (For example, see the command ''semilogy'' in
matlab)}\label{for-each-method-sd-and-cg-for-each-h-plot-the-relative-change-in-the-solution-lhs-of-eq.-3-versus-the-iteration-index-k.-in-the-plot-the-relative-change-in-the-solution-should-be-in-base-10-logarithmic-scale-for-example-see-the-command-semilogy-in-matlab}}

    \begin{center}\rule{0.5\linewidth}{0.5pt}\end{center}

    \hypertarget{answer-1-steepest-descent-sd}{%
\subsection{Answer (1): Steepest Descent
(SD)}\label{answer-1-steepest-descent-sd}}

    \hypertarget{set-of-values-of-the-h}{%
\subsection{Set of values of the h}\label{set-of-values-of-the-h}}

    \begin{tcolorbox}[breakable, size=fbox, boxrule=1pt, pad at break*=1mm,colback=cellbackground, colframe=cellborder]
\prompt{In}{incolor}{5}{\boxspacing}
\begin{Verbatim}[commandchars=\\\{\}]
\PY{n}{H} \PY{o}{=} \PY{p}{[}\PY{l+m+mi}{1}\PY{o}{/}\PY{l+m+mi}{10}\PY{p}{,}\PY{l+m+mi}{1}\PY{o}{/}\PY{l+m+mi}{20}\PY{p}{,}\PY{l+m+mi}{1}\PY{o}{/}\PY{l+m+mi}{40}\PY{p}{]}
\end{Verbatim}
\end{tcolorbox}

    \hypertarget{defining-the-domain-omega-01-times-0-1}{%
\subsection{\texorpdfstring{Defining the domain: \(\Omega\) = {[}0,1{]}
\(\times\) {[}0,
1{]}}{Defining the domain: \textbackslash{}Omega = {[}0,1{]} \textbackslash{}times {[}0, 1{]}}}\label{defining-the-domain-omega-01-times-0-1}}

    \begin{tcolorbox}[breakable, size=fbox, boxrule=1pt, pad at break*=1mm,colback=cellbackground, colframe=cellborder]
\prompt{In}{incolor}{6}{\boxspacing}
\begin{Verbatim}[commandchars=\\\{\}]
\PY{n}{x\PYZus{}0} \PY{o}{=} \PY{l+m+mi}{0}
\end{Verbatim}
\end{tcolorbox}

    \begin{tcolorbox}[breakable, size=fbox, boxrule=1pt, pad at break*=1mm,colback=cellbackground, colframe=cellborder]
\prompt{In}{incolor}{7}{\boxspacing}
\begin{Verbatim}[commandchars=\\\{\}]
\PY{n}{x\PYZus{}l} \PY{o}{=} \PY{l+m+mi}{1}
\end{Verbatim}
\end{tcolorbox}

    \begin{tcolorbox}[breakable, size=fbox, boxrule=1pt, pad at break*=1mm,colback=cellbackground, colframe=cellborder]
\prompt{In}{incolor}{8}{\boxspacing}
\begin{Verbatim}[commandchars=\\\{\}]
\PY{n}{y\PYZus{}0} \PY{o}{=} \PY{l+m+mi}{0}
\end{Verbatim}
\end{tcolorbox}

    \begin{tcolorbox}[breakable, size=fbox, boxrule=1pt, pad at break*=1mm,colback=cellbackground, colframe=cellborder]
\prompt{In}{incolor}{9}{\boxspacing}
\begin{Verbatim}[commandchars=\\\{\}]
\PY{n}{y\PYZus{}l} \PY{o}{=} \PY{l+m+mi}{1}
\end{Verbatim}
\end{tcolorbox}

    \hypertarget{function-to-implement-the-boundary-conditions}{%
\subsection{Function to implement the boundary
conditions}\label{function-to-implement-the-boundary-conditions}}

    \[g(x,y) = \begin{cases}
0 \qquad\qquad\qquad if\:x = 0
\\
y \qquad\qquad\qquad if\:x = 1
\\
(x-1)sin(x) \:\:\:\:\:\:\: if\:y = 0
\\
x(2-x) \qquad\:\:\:\:\:\: if\:y = 1
\end{cases}\]

    \begin{tcolorbox}[breakable, size=fbox, boxrule=1pt, pad at break*=1mm,colback=cellbackground, colframe=cellborder]
\prompt{In}{incolor}{10}{\boxspacing}
\begin{Verbatim}[commandchars=\\\{\}]
\PY{k}{def} \PY{n+nf}{g}\PY{p}{(}\PY{n}{x}\PY{p}{,}\PY{n}{y}\PY{p}{)}\PY{p}{:}
\PY{+w}{    }\PY{l+s+sd}{\PYZdq{}\PYZdq{}\PYZdq{}}
\PY{l+s+sd}{    g(x,y) sets the boundary conditions at based on the coordinates of the node, x and y.}
\PY{l+s+sd}{    If the given node does not lie on the boundary then 0 will be returned.}
\PY{l+s+sd}{    x: x coordinate}
\PY{l+s+sd}{    y: y coordinate}
\PY{l+s+sd}{    \PYZdq{}\PYZdq{}\PYZdq{}}
    \PY{k}{if} \PY{n}{x} \PY{o}{==} \PY{l+m+mi}{0}\PY{p}{:}
        \PY{k}{return} \PY{l+m+mi}{0}
    \PY{k}{if} \PY{n}{x} \PY{o}{==} \PY{l+m+mi}{1}\PY{p}{:}
        \PY{k}{return} \PY{n}{y}
    \PY{k}{if} \PY{n}{y} \PY{o}{==} \PY{l+m+mi}{0}\PY{p}{:}
        \PY{k}{return} \PY{p}{(}\PY{n}{x}\PY{o}{\PYZhy{}}\PY{l+m+mi}{1}\PY{p}{)}\PY{o}{*}\PY{n}{np}\PY{o}{.}\PY{n}{sin}\PY{p}{(}\PY{n}{x}\PY{p}{)}
    \PY{k}{if} \PY{n}{y} \PY{o}{==} \PY{l+m+mi}{1}\PY{p}{:}
        \PY{k}{return} \PY{n}{x}\PY{o}{*}\PY{p}{(}\PY{l+m+mi}{2}\PY{o}{\PYZhy{}}\PY{n}{x}\PY{p}{)}
    \PY{k}{else}\PY{p}{:} 
        \PY{k}{return} \PY{l+m+mi}{0}
\end{Verbatim}
\end{tcolorbox}

    \begin{tcolorbox}[breakable, size=fbox, boxrule=1pt, pad at break*=1mm,colback=cellbackground, colframe=cellborder]
\prompt{In}{incolor}{11}{\boxspacing}
\begin{Verbatim}[commandchars=\\\{\}]
\PY{k}{def} \PY{n+nf}{f}\PY{p}{(}\PY{n}{x}\PY{p}{,}\PY{n}{y}\PY{p}{)}\PY{p}{:}
\PY{+w}{    }\PY{l+s+sd}{\PYZdq{}\PYZdq{}\PYZdq{}}
\PY{l+s+sd}{    f(x,y) evaluated the function f of the question based on the coordinates of the node, x and y.}
\PY{l+s+sd}{    x: x coordinate}
\PY{l+s+sd}{    y: y coordinate}
\PY{l+s+sd}{    \PYZdq{}\PYZdq{}\PYZdq{}}
    \PY{k}{return} \PY{l+m+mi}{0}
\end{Verbatim}
\end{tcolorbox}

    \hypertarget{stopping-criteria}{%
\subsection{Stopping criteria:}\label{stopping-criteria}}

    \[\frac{||u^{(k+1)}-u^{(k)}||_2}{||u^{(k+1)}||_2} < 10^{-8} \tag{3}\]

    \begin{tcolorbox}[breakable, size=fbox, boxrule=1pt, pad at break*=1mm,colback=cellbackground, colframe=cellborder]
\prompt{In}{incolor}{12}{\boxspacing}
\begin{Verbatim}[commandchars=\\\{\}]
\PY{k}{def} \PY{n+nf}{Error\PYZus{}Function}\PY{p}{(}\PY{n}{u\PYZus{}new}\PY{p}{,}\PY{n}{u}\PY{p}{)}\PY{p}{:}
\PY{+w}{    }\PY{l+s+sd}{\PYZdq{}\PYZdq{}\PYZdq{}}
\PY{l+s+sd}{    Error\PYZus{}Function(u\PYZus{}new,u) evaluates the error.}
\PY{l+s+sd}{    u\PYZus{}new: u(k+1)}
\PY{l+s+sd}{    u: u(k)}
\PY{l+s+sd}{    \PYZdq{}\PYZdq{}\PYZdq{}}
    
    \PY{c+c1}{\PYZsh{} Calculating the numerator}
    \PY{n}{temp} \PY{o}{=} \PY{p}{(}\PY{p}{(}\PY{p}{(}\PY{n}{u\PYZus{}new}\PY{o}{.}\PY{n}{flatten}\PY{p}{(}\PY{p}{)}\PY{o}{\PYZhy{}}\PY{n}{u}\PY{o}{.}\PY{n}{flatten}\PY{p}{(}\PY{p}{)}\PY{p}{)}\PY{o}{*}\PY{o}{*}\PY{l+m+mi}{2}\PY{p}{)}\PY{o}{.}\PY{n}{sum}\PY{p}{(}\PY{p}{)}\PY{p}{)}\PY{o}{*}\PY{o}{*}\PY{l+m+mf}{0.5}
    
    \PY{c+c1}{\PYZsh{} Calculating the denominator}
    \PY{n}{temp\PYZus{}1} \PY{o}{=} \PY{p}{(}\PY{p}{(}\PY{p}{(}\PY{n}{u\PYZus{}new}\PY{o}{.}\PY{n}{flatten}\PY{p}{(}\PY{p}{)}\PY{p}{)}\PY{o}{*}\PY{o}{*}\PY{l+m+mi}{2}\PY{p}{)}\PY{o}{.}\PY{n}{sum}\PY{p}{(}\PY{p}{)}\PY{p}{)}\PY{o}{*}\PY{o}{*}\PY{l+m+mf}{0.5}
    
    \PY{c+c1}{\PYZsh{} Calculating the error}
    \PY{n}{temp} \PY{o}{=} \PY{n}{temp}\PY{o}{/}\PY{n}{temp\PYZus{}1}
    
    \PY{k}{return} \PY{n}{temp}
\end{Verbatim}
\end{tcolorbox}

    \begin{tcolorbox}[breakable, size=fbox, boxrule=1pt, pad at break*=1mm,colback=cellbackground, colframe=cellborder]
\prompt{In}{incolor}{13}{\boxspacing}
\begin{Verbatim}[commandchars=\\\{\}]
\PY{n}{Tolerance} \PY{o}{=} \PY{l+m+mf}{1e\PYZhy{}8}
\end{Verbatim}
\end{tcolorbox}

    \hypertarget{a-dictionary-to-store-errors}{%
\subsection{A dictionary to store
errors}\label{a-dictionary-to-store-errors}}

    \begin{tcolorbox}[breakable, size=fbox, boxrule=1pt, pad at break*=1mm,colback=cellbackground, colframe=cellborder]
\prompt{In}{incolor}{14}{\boxspacing}
\begin{Verbatim}[commandchars=\\\{\}]
\PY{n}{Error} \PY{o}{=} \PY{p}{\PYZob{}}\PY{p}{\PYZcb{}}
\end{Verbatim}
\end{tcolorbox}

    \hypertarget{a-dictionary-to-store-u}{%
\subsection{A dictionary to store u}\label{a-dictionary-to-store-u}}

    \begin{tcolorbox}[breakable, size=fbox, boxrule=1pt, pad at break*=1mm,colback=cellbackground, colframe=cellborder]
\prompt{In}{incolor}{15}{\boxspacing}
\begin{Verbatim}[commandchars=\\\{\}]
\PY{n}{U} \PY{o}{=} \PY{p}{\PYZob{}}\PY{p}{\PYZcb{}}
\end{Verbatim}
\end{tcolorbox}

    \hypertarget{a-dictionary-to-store-meshgrids}{%
\subsection{A dictionary to store
meshgrids}\label{a-dictionary-to-store-meshgrids}}

    \begin{tcolorbox}[breakable, size=fbox, boxrule=1pt, pad at break*=1mm,colback=cellbackground, colframe=cellborder]
\prompt{In}{incolor}{16}{\boxspacing}
\begin{Verbatim}[commandchars=\\\{\}]
\PY{n}{X\PYZus{}Y\PYZus{}dict} \PY{o}{=} \PY{p}{\PYZob{}}\PY{p}{\PYZcb{}}
\end{Verbatim}
\end{tcolorbox}

    \hypertarget{function-to-calculate-the-product-of-two-matrices}{%
\subsection{Function to calculate the product of two
matrices}\label{function-to-calculate-the-product-of-two-matrices}}

    \begin{tcolorbox}[breakable, size=fbox, boxrule=1pt, pad at break*=1mm,colback=cellbackground, colframe=cellborder]
\prompt{In}{incolor}{17}{\boxspacing}
\begin{Verbatim}[commandchars=\\\{\}]
\PY{k}{def} \PY{n+nf}{MatMul}\PY{p}{(}\PY{n}{A}\PY{p}{,}\PY{n}{B}\PY{p}{)}\PY{p}{:}
\PY{+w}{    }\PY{l+s+sd}{\PYZdq{}\PYZdq{}\PYZdq{}}
\PY{l+s+sd}{    MatMul(A,B) multiply two matrices A and B of compatible sizes}
\PY{l+s+sd}{    NOTE: This function is a generic matrix multiplication function.}
\PY{l+s+sd}{    \PYZdq{}\PYZdq{}\PYZdq{}}
    
    \PY{c+c1}{\PYZsh{} We are copying A and B because we will be reshaping them}
    \PY{n}{a} \PY{o}{=} \PY{n}{A}\PY{o}{.}\PY{n}{copy}\PY{p}{(}\PY{p}{)}
    \PY{n}{b} \PY{o}{=} \PY{n}{B}\PY{o}{.}\PY{n}{copy}\PY{p}{(}\PY{p}{)}
    
    \PY{c+c1}{\PYZsh{} Checking the dimension of a}
    \PY{k}{if} \PY{n+nb}{len}\PY{p}{(}\PY{n}{a}\PY{o}{.}\PY{n}{shape}\PY{p}{)} \PY{o}{\PYZgt{}} \PY{l+m+mi}{2}\PY{p}{:}
        \PY{n+nb}{print}\PY{p}{(}\PY{l+s+sa}{f}\PY{l+s+s2}{\PYZdq{}}\PY{l+s+s2}{A is NOT a matrix.}\PY{l+s+s2}{\PYZdq{}}\PY{p}{)}
        \PY{n}{sys}\PY{o}{.}\PY{n}{exit}\PY{p}{(}\PY{p}{)}

    \PY{c+c1}{\PYZsh{} Checking the dimension of b}
    \PY{k}{if} \PY{n+nb}{len}\PY{p}{(}\PY{n}{b}\PY{o}{.}\PY{n}{shape}\PY{p}{)} \PY{o}{\PYZgt{}} \PY{l+m+mi}{2}\PY{p}{:}
        \PY{n+nb}{print}\PY{p}{(}\PY{l+s+sa}{f}\PY{l+s+s2}{\PYZdq{}}\PY{l+s+s2}{B is NOT a matrix.}\PY{l+s+s2}{\PYZdq{}}\PY{p}{)}
        \PY{n}{sys}\PY{o}{.}\PY{n}{exit}\PY{p}{(}\PY{p}{)}
            
    \PY{c+c1}{\PYZsh{} a will be I x J}
    \PY{c+c1}{\PYZsh{} b will be J x K}
    
    \PY{n}{I} \PY{o}{=} \PY{n}{a}\PY{o}{.}\PY{n}{shape}\PY{p}{[}\PY{l+m+mi}{0}\PY{p}{]}
    
    \PY{c+c1}{\PYZsh{} To take care of the vectors}
    \PY{k}{if} \PY{n+nb}{len}\PY{p}{(}\PY{n}{a}\PY{o}{.}\PY{n}{shape}\PY{p}{)} \PY{o}{\PYZlt{}} \PY{l+m+mi}{2}\PY{p}{:}
        \PY{n}{J} \PY{o}{=} \PY{l+m+mi}{1}
    \PY{k}{else}\PY{p}{:}
        \PY{n}{J} \PY{o}{=} \PY{n}{a}\PY{o}{.}\PY{n}{shape}\PY{p}{[}\PY{l+m+mi}{1}\PY{p}{]}
    
    \PY{c+c1}{\PYZsh{} To take care of the vectors    }
    \PY{k}{if} \PY{n+nb}{len}\PY{p}{(}\PY{n}{b}\PY{o}{.}\PY{n}{shape}\PY{p}{)} \PY{o}{\PYZlt{}} \PY{l+m+mi}{2}\PY{p}{:}
        \PY{n}{K} \PY{o}{=} \PY{l+m+mi}{1}
    \PY{k}{else}\PY{p}{:}
        \PY{n}{K} \PY{o}{=} \PY{n}{b}\PY{o}{.}\PY{n}{shape}\PY{p}{[}\PY{l+m+mi}{1}\PY{p}{]}
    
    \PY{c+c1}{\PYZsh{} Checking if the dimensions of A and B are compatible for multiplication or NOT.    }
    \PY{k}{if} \PY{n}{J} \PY{o}{!=} \PY{n}{b}\PY{o}{.}\PY{n}{shape}\PY{p}{[}\PY{l+m+mi}{0}\PY{p}{]}\PY{p}{:}
        \PY{n+nb}{print}\PY{p}{(}\PY{l+s+sa}{f}\PY{l+s+s2}{\PYZdq{}}\PY{l+s+s2}{Dimensions of A and B are NOT compatible.}\PY{l+s+s2}{\PYZdq{}}\PY{p}{)}
        \PY{n}{sys}\PY{o}{.}\PY{n}{exit}\PY{p}{(}\PY{p}{)}
    
    \PY{c+c1}{\PYZsh{} Prod matrix will store the results}
    \PY{n}{Prod} \PY{o}{=} \PY{n}{np}\PY{o}{.}\PY{n}{zeros}\PY{p}{(}\PY{p}{(}\PY{n}{I}\PY{p}{,}\PY{n}{K}\PY{p}{)}\PY{p}{)}
    
    \PY{c+c1}{\PYZsh{} a will be I x J}
    \PY{n}{a} \PY{o}{=} \PY{n}{a}\PY{o}{.}\PY{n}{reshape}\PY{p}{(}\PY{n}{I}\PY{p}{,}\PY{n}{J}\PY{p}{)}
    
    \PY{c+c1}{\PYZsh{} b will be J x K}
    \PY{n}{b} \PY{o}{=} \PY{n}{b}\PY{o}{.}\PY{n}{reshape}\PY{p}{(}\PY{n}{J}\PY{p}{,}\PY{n}{K}\PY{p}{)}
    
    \PY{c+c1}{\PYZsh{} Multiplying the matrix A and matrix B}
    \PY{k}{for} \PY{n}{i} \PY{o+ow}{in} \PY{n+nb}{range}\PY{p}{(}\PY{n}{I}\PY{p}{)}\PY{p}{:}
        \PY{k}{for} \PY{n}{j} \PY{o+ow}{in} \PY{n+nb}{range}\PY{p}{(}\PY{n}{J}\PY{p}{)}\PY{p}{:}
            \PY{k}{for} \PY{n}{k} \PY{o+ow}{in} \PY{n+nb}{range}\PY{p}{(}\PY{n}{K}\PY{p}{)}\PY{p}{:}
                \PY{n}{Prod}\PY{p}{[}\PY{n}{i}\PY{p}{]}\PY{p}{[}\PY{n}{k}\PY{p}{]} \PY{o}{=} \PY{n}{Prod}\PY{p}{[}\PY{n}{i}\PY{p}{]}\PY{p}{[}\PY{n}{k}\PY{p}{]} \PY{o}{+} \PY{p}{(}\PY{n}{a}\PY{p}{[}\PY{n}{i}\PY{p}{]}\PY{p}{[}\PY{n}{j}\PY{p}{]}\PY{o}{*}\PY{n}{b}\PY{p}{[}\PY{n}{j}\PY{p}{]}\PY{p}{[}\PY{n}{k}\PY{p}{]}\PY{p}{)}
    
    \PY{c+c1}{\PYZsh{} Returning the matrix product}
    \PY{k}{return} \PY{n}{Prod}
\end{Verbatim}
\end{tcolorbox}

    \hypertarget{discretization}{%
\subsection{Discretization:}\label{discretization}}

    \[-\frac{\partial^2 u}{\partial x^2 } -\frac{\partial^2 u}{\partial y^2 } = f\qquad in \qquad \Omega\]

    \[\frac{\partial^2 u}{\partial x^2 } + \frac{\partial^2 u}{\partial y^2 } = -f\qquad in \qquad \Omega\]

    \[\Delta x = \Delta y = h\]

    \[\frac{u_{(i+1,j)} - 2u_{(i,j)} + u_{(i-1,j)}}{h^2 } + \frac{u_{(i,j+1)} - 2u_{(i,j)} + u_{(i,j-1)}}{h^2 } = f_{(i,j)}\]

    or,
\[\frac{u_{(i+1,j)} - 2u_{(i,j)} + u_{(i-1,j)} + u_{(i,j+1)} - 2u_{(i,j)} + u_{(i,j-1)}}{h^2 } = f_{(i,j)}\]

    or,
\[\frac{u_{(i+1,j)} + u_{(i,j+1)} - 4u_{(i,j)} + u_{(i-1,j)} + u_{(i,j-1)}}{h^2 } = f_{(i,j)}\]

    or,
\[u_{(i+1,j)} + u_{(i,j+1)} - 4u_{(i,j)} + u_{(i-1,j)} + u_{(i,j-1)} = h^2f_{(i,j)}\]

    or,
\[u_{(i+1,j)} + u_{(i,j+1)} - 4u_{(i,j)} + u_{(i,j-1)} + u_{(i-1,j)} = h^2f_{(i,j)}\]

    Let, \[I = I(i,j) = (n_x \times i) + j\]

    So, \[u_{(i,j)} = u_{I}\]

    So, \[I(i,j-1) = (n \times i) + j - 1\]

    or, \[I(i,j-1) = I - 1\]

    So, \[u_{(i,j-1)} = u_{I-1}\]

    Also, \[I(i,j+1) = (n_x \times i) + j + 1\]

    or, \[I(i,j+1) = I + 1\]

    So, \[u_{(i,j+1)} = u_{I+1}\]

    Also, \[I(i-1,j) = (n_x \times (i-1)) + j\]

    or, \[I(i-1,j) = I - n_x\]

    So, \[u_{(i-1,j)} = u_{I-n_x}\]

    Also, \[I(i+1,j) = (n_x \times (i+1)) + j\]

    or, \[I(i+1,j) = I + n_x\]

    So, \[u_{(i+1,j)} = u_{I + n_x}\]

    So, \[u_{I+n_x} + u_{I+1} - 4u_{I} + u_{I-1} + u_{I-n_x} = h^2f_{I}\]

    \hypertarget{function-to-calculate-the-product-of-a-and-any-vector}{%
\subsection{Function to calculate the product of A and any
vector}\label{function-to-calculate-the-product-of-a-and-any-vector}}

    \begin{tcolorbox}[breakable, size=fbox, boxrule=1pt, pad at break*=1mm,colback=cellbackground, colframe=cellborder]
\prompt{In}{incolor}{18}{\boxspacing}
\begin{Verbatim}[commandchars=\\\{\}]
\PY{k}{def} \PY{n+nf}{MatMul\PYZus{}A\PYZus{}Vector}\PY{p}{(}\PY{n}{b}\PY{p}{,}\PY{n}{n}\PY{p}{,}\PY{n}{n\PYZus{}x}\PY{p}{)}\PY{p}{:}
\PY{+w}{    }\PY{l+s+sd}{\PYZdq{}\PYZdq{}\PYZdq{}}
\PY{l+s+sd}{    MatMul\PYZus{}A\PYZus{}Vector(b,n,n\PYZus{}x) multiply matrix A for this particular case and a generic vector b of compatible sizes.}
\PY{l+s+sd}{    NOTE: We are exploiting the sparse nature of the matrix A.}
\PY{l+s+sd}{    NOTE: This function is specific to this problem only and NOT a generic matrix multiplication function.}
\PY{l+s+sd}{    \PYZdq{}\PYZdq{}\PYZdq{}}
    
    \PY{c+c1}{\PYZsh{} A will be I x J}
    \PY{c+c1}{\PYZsh{} b will be J x K}
    
    \PY{n}{I} \PY{o}{=} \PY{n}{n}
    
    \PY{c+c1}{\PYZsh{} Checking the dimension of b}
    \PY{k}{if} \PY{n+nb}{len}\PY{p}{(}\PY{n}{b}\PY{o}{.}\PY{n}{shape}\PY{p}{)} \PY{o}{!=} \PY{l+m+mi}{2}\PY{p}{:}
        \PY{k}{if} \PY{n}{b}\PY{o}{.}\PY{n}{shape}\PY{p}{[}\PY{l+m+mi}{1}\PY{p}{]} \PY{o}{!=} \PY{l+m+mi}{1}\PY{p}{:}
            \PY{n+nb}{print}\PY{p}{(}\PY{l+s+sa}{f}\PY{l+s+s2}{\PYZdq{}}\PY{l+s+s2}{b is NOT a vector.}\PY{l+s+s2}{\PYZdq{}}\PY{p}{)}
            \PY{n}{sys}\PY{o}{.}\PY{n}{exit}\PY{p}{(}\PY{p}{)}
            
    \PY{c+c1}{\PYZsh{} Checking if the dimensions of A and b are compatible for multiplication or NOT.            }
    \PY{k}{if} \PY{n}{b}\PY{o}{.}\PY{n}{shape}\PY{p}{[}\PY{l+m+mi}{0}\PY{p}{]} \PY{o}{!=} \PY{n}{n}\PY{p}{:}
        \PY{n+nb}{print}\PY{p}{(}\PY{l+s+sa}{f}\PY{l+s+s2}{\PYZdq{}}\PY{l+s+s2}{Dimensions of A and b are not compatible.}\PY{l+s+s2}{\PYZdq{}}\PY{p}{)}
        \PY{n}{sys}\PY{o}{.}\PY{n}{exit}\PY{p}{(}\PY{p}{)}
    
    \PY{n}{J} \PY{o}{=} \PY{n}{n}
    
    \PY{n}{K} \PY{o}{=} \PY{l+m+mi}{1}
    
    \PY{c+c1}{\PYZsh{} Prod matrix will store the results}
    \PY{n}{Prod} \PY{o}{=} \PY{n}{np}\PY{o}{.}\PY{n}{zeros}\PY{p}{(}\PY{p}{(}\PY{n}{I}\PY{p}{,}\PY{n}{K}\PY{p}{)}\PY{p}{)}
    
    \PY{c+c1}{\PYZsh{} b will be J x K}
    \PY{n}{b} \PY{o}{=} \PY{n}{b}\PY{o}{.}\PY{n}{reshape}\PY{p}{(}\PY{n}{J}\PY{p}{,}\PY{n}{K}\PY{p}{)}
    
    \PY{c+c1}{\PYZsh{} Doing the matrix multiplication by only multiplying the non\PYZhy{}zero elements}

    \PY{k}{for} \PY{n}{i} \PY{o+ow}{in} \PY{n+nb}{range}\PY{p}{(}\PY{l+m+mi}{0}\PY{p}{,}\PY{n}{n\PYZus{}x}\PY{p}{)}\PY{p}{:}
        \PY{n}{Prod}\PY{p}{[}\PY{n}{i}\PY{p}{]}\PY{p}{[}\PY{l+m+mi}{0}\PY{p}{]} \PY{o}{=} \PY{n}{b}\PY{p}{[}\PY{n}{i}\PY{p}{]}\PY{p}{[}\PY{l+m+mi}{0}\PY{p}{]}
        
    \PY{k}{for} \PY{n}{i} \PY{o+ow}{in} \PY{n+nb}{range}\PY{p}{(}\PY{n}{n\PYZus{}x}\PY{p}{,}\PY{n}{I}\PY{o}{\PYZhy{}}\PY{n}{n\PYZus{}x}\PY{p}{)}\PY{p}{:}
        \PY{k}{if} \PY{p}{(}\PY{n}{i}\PY{o}{\PYZpc{}}\PY{k}{n\PYZus{}x} == 0) or ((i+1)\PYZpc{}n\PYZus{}x == 0):
            \PY{n}{Prod}\PY{p}{[}\PY{n}{i}\PY{p}{]}\PY{p}{[}\PY{l+m+mi}{0}\PY{p}{]} \PY{o}{=} \PY{n}{b}\PY{p}{[}\PY{n}{i}\PY{p}{]}\PY{p}{[}\PY{l+m+mi}{0}\PY{p}{]}
        \PY{k}{else}\PY{p}{:}
            \PY{n}{Prod}\PY{p}{[}\PY{n}{i}\PY{p}{]}\PY{p}{[}\PY{l+m+mi}{0}\PY{p}{]} \PY{o}{=} \PY{n}{Prod}\PY{p}{[}\PY{n}{i}\PY{p}{]}\PY{p}{[}\PY{l+m+mi}{0}\PY{p}{]} \PY{o}{+} \PY{p}{(}\PY{o}{\PYZhy{}}\PY{l+m+mi}{4}\PY{o}{*}\PY{n}{b}\PY{p}{[}\PY{n}{i}\PY{p}{]}\PY{p}{[}\PY{l+m+mi}{0}\PY{p}{]}\PY{p}{)}
            \PY{n}{Prod}\PY{p}{[}\PY{n}{i}\PY{p}{]}\PY{p}{[}\PY{l+m+mi}{0}\PY{p}{]} \PY{o}{=} \PY{n}{Prod}\PY{p}{[}\PY{n}{i}\PY{p}{]}\PY{p}{[}\PY{l+m+mi}{0}\PY{p}{]} \PY{o}{+} \PY{p}{(}\PY{n}{b}\PY{p}{[}\PY{n}{i}\PY{o}{+}\PY{l+m+mi}{1}\PY{p}{]}\PY{p}{[}\PY{l+m+mi}{0}\PY{p}{]}\PY{p}{)}
            \PY{n}{Prod}\PY{p}{[}\PY{n}{i}\PY{p}{]}\PY{p}{[}\PY{l+m+mi}{0}\PY{p}{]} \PY{o}{=} \PY{n}{Prod}\PY{p}{[}\PY{n}{i}\PY{p}{]}\PY{p}{[}\PY{l+m+mi}{0}\PY{p}{]} \PY{o}{+} \PY{p}{(}\PY{n}{b}\PY{p}{[}\PY{n}{i}\PY{o}{\PYZhy{}}\PY{l+m+mi}{1}\PY{p}{]}\PY{p}{[}\PY{l+m+mi}{0}\PY{p}{]}\PY{p}{)}
            \PY{n}{Prod}\PY{p}{[}\PY{n}{i}\PY{p}{]}\PY{p}{[}\PY{l+m+mi}{0}\PY{p}{]} \PY{o}{=} \PY{n}{Prod}\PY{p}{[}\PY{n}{i}\PY{p}{]}\PY{p}{[}\PY{l+m+mi}{0}\PY{p}{]} \PY{o}{+} \PY{p}{(}\PY{n}{b}\PY{p}{[}\PY{n}{i}\PY{o}{+}\PY{n}{n\PYZus{}x}\PY{p}{]}\PY{p}{[}\PY{l+m+mi}{0}\PY{p}{]}\PY{p}{)}
            \PY{n}{Prod}\PY{p}{[}\PY{n}{i}\PY{p}{]}\PY{p}{[}\PY{l+m+mi}{0}\PY{p}{]} \PY{o}{=} \PY{n}{Prod}\PY{p}{[}\PY{n}{i}\PY{p}{]}\PY{p}{[}\PY{l+m+mi}{0}\PY{p}{]} \PY{o}{+} \PY{p}{(}\PY{n}{b}\PY{p}{[}\PY{n}{i}\PY{o}{\PYZhy{}}\PY{n}{n\PYZus{}x}\PY{p}{]}\PY{p}{[}\PY{l+m+mi}{0}\PY{p}{]}\PY{p}{)}
    
    \PY{k}{for} \PY{n}{i} \PY{o+ow}{in} \PY{n+nb}{range}\PY{p}{(}\PY{n}{I}\PY{o}{\PYZhy{}}\PY{n}{n\PYZus{}x}\PY{p}{,}\PY{n}{I}\PY{p}{)}\PY{p}{:}
        \PY{n}{Prod}\PY{p}{[}\PY{n}{i}\PY{p}{]}\PY{p}{[}\PY{l+m+mi}{0}\PY{p}{]} \PY{o}{=} \PY{n}{b}\PY{p}{[}\PY{n}{i}\PY{p}{]}\PY{p}{[}\PY{l+m+mi}{0}\PY{p}{]}
    
    \PY{c+c1}{\PYZsh{} Returning the matrix product}
    \PY{k}{return} \PY{n}{Prod}
\end{Verbatim}
\end{tcolorbox}

    \hypertarget{steepest-descent-sd-algorithm-for-au-b}{%
\subsection{Steepest Descent (SD) Algorithm for Au =
b}\label{steepest-descent-sd-algorithm-for-au-b}}

    \(r^{(0)} \leftarrow (b - Au^{(0)})\)

    \(P^{(0)} \leftarrow r^{(0)}\)

    do until convergence:

    \(\qquad q^{(k)} \leftarrow AP^{(k)}\)

    \(\qquad \alpha^{(k)} \leftarrow \frac{(P^{(k)})^Tr^{(k)}}{(P^{(k)})^Tq^{(k)}}\)

    \(\qquad u^{(k+1)} \leftarrow (u^{(k)} + \alpha^{(k)}P^{(k)})\)

    \(\qquad r^{(k+1)} \leftarrow (r^{(k)} + \alpha^{(k)}q^{(k)})\)

    \(\qquad P^{(k+1)} \leftarrow r^{(k+1)}\)

    \(\qquad k \leftarrow (k+1)\)

    \begin{tcolorbox}[breakable, size=fbox, boxrule=1pt, pad at break*=1mm,colback=cellbackground, colframe=cellborder]
\prompt{In}{incolor}{19}{\boxspacing}
\begin{Verbatim}[commandchars=\\\{\}]
\PY{k}{def} \PY{n+nf}{SD\PYZus{}Solver}\PY{p}{(}\PY{n}{r}\PY{p}{,}\PY{n}{u}\PY{p}{,}\PY{n}{U}\PY{p}{,}\PY{n}{Error}\PY{p}{,}\PY{n}{Tolerance}\PY{p}{,}\PY{n}{f}\PY{p}{,}\PY{n}{x}\PY{p}{,}\PY{n}{y}\PY{p}{,}\PY{n}{h}\PY{p}{,}\PY{n}{n}\PY{p}{,}\PY{n}{n\PYZus{}x}\PY{p}{,}\PY{n}{n\PYZus{}y}\PY{p}{)}\PY{p}{:}
\PY{+w}{    }\PY{l+s+sd}{\PYZdq{}\PYZdq{}\PYZdq{}}
\PY{l+s+sd}{    SD\PYZus{}Solver(A,b,u,U,Error,Tolerance,f,x,y,h,n\PYZus{}x,n\PYZus{}y): Solves the given poisson equation using the Steepest Descent (SD) method}
\PY{l+s+sd}{    r: r is the residual but b is stored in r, where Au = b and r = b\PYZhy{}(Au)}
\PY{l+s+sd}{    u: Initial value of u in the computational domain}
\PY{l+s+sd}{    U: Dictionary to store u for different values of h}
\PY{l+s+sd}{    Error: Dictionary to store error at each iterations for different values of h}
\PY{l+s+sd}{    Tolerance: Stopping Criteria}
\PY{l+s+sd}{    f: RHS of the equation}
\PY{l+s+sd}{    x: X Meshgrid}
\PY{l+s+sd}{    y: Y Meshgrid}
\PY{l+s+sd}{    h: Grid Spacing}
\PY{l+s+sd}{    n: Total number of grid points}
\PY{l+s+sd}{    n\PYZus{}x: Number of the grid points in the x\PYZhy{}directions}
\PY{l+s+sd}{    n\PYZus{}y: Number of the grid points in the y\PYZhy{}directions}
\PY{l+s+sd}{    \PYZdq{}\PYZdq{}\PYZdq{}}
    
    \PY{c+c1}{\PYZsh{} Intializing the tolerance achieved}
    \PY{n}{temp} \PY{o}{=} \PY{n}{Tolerance} \PY{o}{+} \PY{l+m+mi}{1}

    \PY{c+c1}{\PYZsh{} Error over iterations for some particular h}
    \PY{n}{error} \PY{o}{=} \PY{p}{[}\PY{p}{]}

    \PY{c+c1}{\PYZsh{} Steepest Descent (SD) algorithm}
    
    \PY{c+c1}{\PYZsh{} Calculating the residual}
    \PY{c+c1}{\PYZsh{} r = b\PYZhy{}(Au)}
    \PY{n}{r} \PY{o}{=} \PY{n}{r}\PY{o}{\PYZhy{}}\PY{n}{MatMul\PYZus{}A\PYZus{}Vector}\PY{p}{(}\PY{n}{u}\PY{p}{,}\PY{n}{n}\PY{p}{,}\PY{n}{n\PYZus{}x}\PY{p}{)}
    
    \PY{c+c1}{\PYZsh{} While loop until the stopping criteria is met}
    \PY{k}{while} \PY{n}{temp} \PY{o}{\PYZgt{}}\PY{o}{=} \PY{n}{Tolerance}\PY{p}{:}
        \PY{n}{u\PYZus{}old} \PY{o}{=} \PY{n}{u}\PY{o}{.}\PY{n}{copy}\PY{p}{(}\PY{p}{)}
        \PY{n}{P} \PY{o}{=} \PY{n}{r}\PY{o}{.}\PY{n}{copy}\PY{p}{(}\PY{p}{)}
        \PY{c+c1}{\PYZsh{} q = AP}
        \PY{n}{q} \PY{o}{=} \PY{n}{MatMul\PYZus{}A\PYZus{}Vector}\PY{p}{(}\PY{n}{P}\PY{p}{,}\PY{n}{n}\PY{p}{,}\PY{n}{n\PYZus{}x}\PY{p}{)}
        \PY{n}{alpha} \PY{o}{=} \PY{n}{MatMul}\PY{p}{(}\PY{n}{P}\PY{o}{.}\PY{n}{T}\PY{p}{,}\PY{n}{r}\PY{p}{)}\PY{o}{/}\PY{n}{MatMul}\PY{p}{(}\PY{n}{P}\PY{o}{.}\PY{n}{T}\PY{p}{,}\PY{n}{q}\PY{p}{)}
        \PY{n}{u} \PY{o}{=} \PY{n}{u} \PY{o}{+} \PY{p}{(}\PY{n}{alpha}\PY{o}{*}\PY{n}{P}\PY{p}{)}
        \PY{n}{r} \PY{o}{=} \PY{n}{r} \PY{o}{\PYZhy{}} \PY{p}{(}\PY{n}{alpha}\PY{o}{*}\PY{n}{q}\PY{p}{)}
        
        \PY{c+c1}{\PYZsh{} Calculating the relative error}
        \PY{n}{temp} \PY{o}{=} \PY{n}{Error\PYZus{}Function}\PY{p}{(}\PY{n}{u}\PY{p}{,}\PY{n}{u\PYZus{}old}\PY{p}{)}
        
        \PY{c+c1}{\PYZsh{} Storing the errors corresponding to each iteration}
        \PY{n}{error}\PY{o}{.}\PY{n}{append}\PY{p}{(}\PY{n}{temp}\PY{p}{)}
        
    \PY{c+c1}{\PYZsh{} Storing u in the dictionary for some particular h}
    \PY{n}{U}\PY{p}{[}\PY{n}{h}\PY{p}{]} \PY{o}{=} \PY{n}{u\PYZus{}old}\PY{o}{.}\PY{n}{reshape}\PY{p}{(}\PY{p}{(}\PY{n}{n\PYZus{}x}\PY{p}{,}\PY{n}{n\PYZus{}y}\PY{p}{)}\PY{p}{)}
    
    \PY{c+c1}{\PYZsh{} Storing error in the dictionary for some particular h}
    \PY{n}{Error}\PY{p}{[}\PY{n}{h}\PY{p}{]} \PY{o}{=} \PY{n}{error}
\end{Verbatim}
\end{tcolorbox}

    \begin{tcolorbox}[breakable, size=fbox, boxrule=1pt, pad at break*=1mm,colback=cellbackground, colframe=cellborder]
\prompt{In}{incolor}{20}{\boxspacing}
\begin{Verbatim}[commandchars=\\\{\}]
\PY{n}{X\PYZus{}Y\PYZus{}dict}
\end{Verbatim}
\end{tcolorbox}

            \begin{tcolorbox}[breakable, size=fbox, boxrule=.5pt, pad at break*=1mm, opacityfill=0]
\prompt{Out}{outcolor}{20}{\boxspacing}
\begin{Verbatim}[commandchars=\\\{\}]
\{\}
\end{Verbatim}
\end{tcolorbox}
        
    \begin{tcolorbox}[breakable, size=fbox, boxrule=1pt, pad at break*=1mm,colback=cellbackground, colframe=cellborder]
\prompt{In}{incolor}{21}{\boxspacing}
\begin{Verbatim}[commandchars=\\\{\}]
\PY{c+c1}{\PYZsh{} Iterating over different values of h (Mesh Interval)}
\PY{k}{for} \PY{n}{h} \PY{o+ow}{in} \PY{n}{H}\PY{p}{:}
    
    \PY{c+c1}{\PYZsh{} Mesh intervals}
    \PY{n}{dx} \PY{o}{=} \PY{n}{h}
    \PY{n}{dy} \PY{o}{=} \PY{n}{h}

    \PY{c+c1}{\PYZsh{} Number of grid points in x\PYZhy{}directions}
    \PY{n}{n\PYZus{}x} \PY{o}{=} \PY{n+nb}{int}\PY{p}{(}\PY{p}{(}\PY{n}{x\PYZus{}l}\PY{o}{\PYZhy{}}\PY{n}{x\PYZus{}0}\PY{p}{)}\PY{o}{/}\PY{n}{h} \PY{o}{+} \PY{l+m+mi}{1}\PY{p}{)}
    
    \PY{c+c1}{\PYZsh{} Number of grid points in y\PYZhy{}directions}
    \PY{n}{n\PYZus{}y} \PY{o}{=} \PY{n+nb}{int}\PY{p}{(}\PY{p}{(}\PY{n}{y\PYZus{}l}\PY{o}{\PYZhy{}}\PY{n}{y\PYZus{}0}\PY{p}{)}\PY{o}{/}\PY{n}{h} \PY{o}{+} \PY{l+m+mi}{1}\PY{p}{)}

    \PY{c+c1}{\PYZsh{} Total number of grid points}
    \PY{n}{n} \PY{o}{=} \PY{n}{n\PYZus{}x}\PY{o}{*}\PY{n}{n\PYZus{}y}

    \PY{c+c1}{\PYZsh{} A matrix to store grid numbering associated with a particular h}
    \PY{n}{I} \PY{o}{=} \PY{n}{np}\PY{o}{.}\PY{n}{zeros}\PY{p}{(}\PY{p}{(}\PY{n}{n\PYZus{}x}\PY{p}{,}\PY{n}{n\PYZus{}y}\PY{p}{)}\PY{p}{,}\PY{n}{dtype} \PY{o}{=} \PY{n}{np}\PY{o}{.}\PY{n}{int32}\PY{p}{)}

    \PY{c+c1}{\PYZsh{} A matrix to store the solution associated with a particular h}
    \PY{n}{u} \PY{o}{=} \PY{n}{np}\PY{o}{.}\PY{n}{zeros}\PY{p}{(}\PY{p}{(}\PY{n}{n\PYZus{}x}\PY{p}{,}\PY{n}{n\PYZus{}y}\PY{p}{)}\PY{p}{)}

    \PY{c+c1}{\PYZsh{} Creating a meshgrid}
    \PY{n}{x} \PY{o}{=} \PY{n}{np}\PY{o}{.}\PY{n}{arange}\PY{p}{(}\PY{n}{x\PYZus{}0}\PY{p}{,}\PY{n}{x\PYZus{}l}\PY{o}{+}\PY{n}{dx}\PY{p}{,}\PY{n}{dx}\PY{p}{)}
    \PY{n}{y} \PY{o}{=} \PY{n}{np}\PY{o}{.}\PY{n}{arange}\PY{p}{(}\PY{n}{y\PYZus{}0}\PY{p}{,}\PY{n}{y\PYZus{}l}\PY{o}{+}\PY{n}{dy}\PY{p}{,}\PY{n}{dy}\PY{p}{)}
    \PY{n}{X}\PY{p}{,}\PY{n}{Y} \PY{o}{=} \PY{n}{np}\PY{o}{.}\PY{n}{meshgrid}\PY{p}{(}\PY{n}{x}\PY{p}{,}\PY{n}{y}\PY{p}{,}\PY{n}{indexing} \PY{o}{=} \PY{l+s+s2}{\PYZdq{}}\PY{l+s+s2}{ij}\PY{l+s+s2}{\PYZdq{}}\PY{p}{)}

    \PY{c+c1}{\PYZsh{} Storing meshgrids}
    \PY{n}{X\PYZus{}Y\PYZus{}dict}\PY{p}{[}\PY{n}{h}\PY{p}{]} \PY{o}{=} \PY{p}{(}\PY{n}{X}\PY{p}{,}\PY{n}{Y}\PY{p}{)}

    \PY{c+c1}{\PYZsh{} Applying the boundary conditions and storing the grid numbers}
    \PY{k}{for} \PY{n}{i} \PY{o+ow}{in} \PY{n+nb}{range}\PY{p}{(}\PY{n}{u}\PY{o}{.}\PY{n}{shape}\PY{p}{[}\PY{l+m+mi}{0}\PY{p}{]}\PY{p}{)}\PY{p}{:}
        \PY{k}{for} \PY{n}{j} \PY{o+ow}{in} \PY{n+nb}{range}\PY{p}{(}\PY{n}{u}\PY{o}{.}\PY{n}{shape}\PY{p}{[}\PY{l+m+mi}{1}\PY{p}{]}\PY{p}{)}\PY{p}{:}
            
            \PY{c+c1}{\PYZsh{} Applying the boundary conditions}
            \PY{n}{u}\PY{p}{[}\PY{n}{i}\PY{p}{,}\PY{n}{j}\PY{p}{]} \PY{o}{=} \PY{n}{g}\PY{p}{(}\PY{n}{x}\PY{p}{[}\PY{n}{i}\PY{p}{]}\PY{p}{,}\PY{n}{y}\PY{p}{[}\PY{n}{j}\PY{p}{]}\PY{p}{)}
            
            \PY{c+c1}{\PYZsh{} Storing the grid numbers}
            \PY{n}{I}\PY{p}{[}\PY{n}{i}\PY{p}{,}\PY{n}{j}\PY{p}{]} \PY{o}{=} \PY{n+nb}{int}\PY{p}{(}\PY{p}{(}\PY{n}{n\PYZus{}x}\PY{o}{*}\PY{n}{i}\PY{p}{)}\PY{o}{+}\PY{n}{j}\PY{p}{)}

    \PY{c+c1}{\PYZsh{} Storing u as a vector}
    \PY{n}{u\PYZus{}linear} \PY{o}{=} \PY{n}{u}\PY{o}{.}\PY{n}{flatten}\PY{p}{(}\PY{p}{)}
    
    \PY{c+c1}{\PYZsh{} Treating vector as a matrix}
    \PY{n}{u\PYZus{}linear} \PY{o}{=} \PY{n}{u\PYZus{}linear}\PY{o}{.}\PY{n}{reshape}\PY{p}{(}\PY{n}{u\PYZus{}linear}\PY{o}{.}\PY{n}{shape}\PY{p}{[}\PY{l+m+mi}{0}\PY{p}{]}\PY{p}{,}\PY{l+m+mi}{1}\PY{p}{)}

    \PY{c+c1}{\PYZsh{} RHS vector stored as matrix}
    \PY{n}{b} \PY{o}{=} \PY{n}{np}\PY{o}{.}\PY{n}{zeros}\PY{p}{(}\PY{p}{(}\PY{n}{n}\PY{p}{,}\PY{l+m+mi}{1}\PY{p}{)}\PY{p}{)}

    \PY{c+c1}{\PYZsh{} Applying the boundary conditions}
    \PY{k}{for} \PY{n}{i} \PY{o+ow}{in} \PY{n+nb}{range}\PY{p}{(}\PY{n}{u}\PY{o}{.}\PY{n}{shape}\PY{p}{[}\PY{l+m+mi}{0}\PY{p}{]}\PY{p}{)}\PY{p}{:}
        \PY{k}{for} \PY{n}{j} \PY{o+ow}{in} \PY{n+nb}{range}\PY{p}{(}\PY{n}{u}\PY{o}{.}\PY{n}{shape}\PY{p}{[}\PY{l+m+mi}{1}\PY{p}{]}\PY{p}{)}\PY{p}{:}
            \PY{k}{if} \PY{n}{i} \PY{o}{==} \PY{l+m+mi}{0}\PY{p}{:} 
                \PY{n}{b}\PY{p}{[}\PY{n}{I}\PY{p}{[}\PY{n}{i}\PY{p}{,}\PY{n}{j}\PY{p}{]}\PY{p}{,}\PY{l+m+mi}{0}\PY{p}{]} \PY{o}{=} \PY{n}{u\PYZus{}linear}\PY{p}{[}\PY{n}{I}\PY{p}{[}\PY{n}{i}\PY{p}{,}\PY{n}{j}\PY{p}{]}\PY{p}{]} \PY{o}{+} \PY{p}{(}\PY{p}{(}\PY{n}{h}\PY{o}{*}\PY{o}{*}\PY{l+m+mi}{2}\PY{p}{)}\PY{o}{*}\PY{n}{f}\PY{p}{(}\PY{n}{x}\PY{p}{[}\PY{n}{i}\PY{p}{]}\PY{p}{,}\PY{n}{y}\PY{p}{[}\PY{n}{j}\PY{p}{]}\PY{p}{)}\PY{p}{)}
            \PY{k}{if} \PY{n}{i} \PY{o}{==} \PY{p}{(}\PY{n}{u}\PY{o}{.}\PY{n}{shape}\PY{p}{[}\PY{l+m+mi}{0}\PY{p}{]}\PY{o}{\PYZhy{}}\PY{l+m+mi}{1}\PY{p}{)}\PY{p}{:}
                \PY{n}{b}\PY{p}{[}\PY{n}{I}\PY{p}{[}\PY{n}{i}\PY{p}{,}\PY{n}{j}\PY{p}{]}\PY{p}{,}\PY{l+m+mi}{0}\PY{p}{]} \PY{o}{=} \PY{n}{u\PYZus{}linear}\PY{p}{[}\PY{n}{I}\PY{p}{[}\PY{n}{i}\PY{p}{,}\PY{n}{j}\PY{p}{]}\PY{p}{]} \PY{o}{+} \PY{p}{(}\PY{p}{(}\PY{n}{h}\PY{o}{*}\PY{o}{*}\PY{l+m+mi}{2}\PY{p}{)}\PY{o}{*}\PY{n}{f}\PY{p}{(}\PY{n}{x}\PY{p}{[}\PY{n}{i}\PY{p}{]}\PY{p}{,}\PY{n}{y}\PY{p}{[}\PY{n}{j}\PY{p}{]}\PY{p}{)}\PY{p}{)}
            \PY{k}{if} \PY{n}{j} \PY{o}{==} \PY{l+m+mi}{0}\PY{p}{:} 
                \PY{n}{b}\PY{p}{[}\PY{n}{I}\PY{p}{[}\PY{n}{i}\PY{p}{,}\PY{n}{j}\PY{p}{]}\PY{p}{,}\PY{l+m+mi}{0}\PY{p}{]} \PY{o}{=} \PY{n}{u\PYZus{}linear}\PY{p}{[}\PY{n}{I}\PY{p}{[}\PY{n}{i}\PY{p}{,}\PY{n}{j}\PY{p}{]}\PY{p}{]} \PY{o}{+} \PY{p}{(}\PY{p}{(}\PY{n}{h}\PY{o}{*}\PY{o}{*}\PY{l+m+mi}{2}\PY{p}{)}\PY{o}{*}\PY{n}{f}\PY{p}{(}\PY{n}{x}\PY{p}{[}\PY{n}{i}\PY{p}{]}\PY{p}{,}\PY{n}{y}\PY{p}{[}\PY{n}{j}\PY{p}{]}\PY{p}{)}\PY{p}{)}
            \PY{k}{if} \PY{n}{j} \PY{o}{==} \PY{p}{(}\PY{n}{u}\PY{o}{.}\PY{n}{shape}\PY{p}{[}\PY{l+m+mi}{1}\PY{p}{]}\PY{o}{\PYZhy{}}\PY{l+m+mi}{1}\PY{p}{)}\PY{p}{:}
                \PY{n}{b}\PY{p}{[}\PY{n}{I}\PY{p}{[}\PY{n}{i}\PY{p}{,}\PY{n}{j}\PY{p}{]}\PY{p}{,}\PY{l+m+mi}{0}\PY{p}{]} \PY{o}{=} \PY{n}{u\PYZus{}linear}\PY{p}{[}\PY{n}{I}\PY{p}{[}\PY{n}{i}\PY{p}{,}\PY{n}{j}\PY{p}{]}\PY{p}{]} \PY{o}{+} \PY{p}{(}\PY{p}{(}\PY{n}{h}\PY{o}{*}\PY{o}{*}\PY{l+m+mi}{2}\PY{p}{)}\PY{o}{*}\PY{n}{f}\PY{p}{(}\PY{n}{x}\PY{p}{[}\PY{n}{i}\PY{p}{]}\PY{p}{,}\PY{n}{y}\PY{p}{[}\PY{n}{j}\PY{p}{]}\PY{p}{)}\PY{p}{)}
            \PY{n}{b}\PY{p}{[}\PY{n}{I}\PY{p}{[}\PY{n}{i}\PY{p}{,}\PY{n}{j}\PY{p}{]}\PY{p}{,}\PY{l+m+mi}{0}\PY{p}{]} \PY{o}{=}  \PY{n}{b}\PY{p}{[}\PY{n}{I}\PY{p}{[}\PY{n}{i}\PY{p}{,}\PY{n}{j}\PY{p}{]}\PY{p}{,}\PY{l+m+mi}{0}\PY{p}{]} \PY{o}{+} \PY{p}{(}\PY{p}{(}\PY{n}{h}\PY{o}{*}\PY{o}{*}\PY{l+m+mi}{2}\PY{p}{)}\PY{o}{*}\PY{n}{f}\PY{p}{(}\PY{n}{x}\PY{p}{[}\PY{n}{i}\PY{p}{]}\PY{p}{,}\PY{n}{y}\PY{p}{[}\PY{n}{j}\PY{p}{]}\PY{p}{)}\PY{p}{)}

    \PY{c+c1}{\PYZsh{} Intializing the u at the current iteraion with zeros}
    \PY{n}{u} \PY{o}{=} \PY{n}{np}\PY{o}{.}\PY{n}{zeros}\PY{p}{(}\PY{p}{(}\PY{p}{(}\PY{n+nb}{int}\PY{p}{(}\PY{p}{(}\PY{n}{x\PYZus{}l}\PY{o}{\PYZhy{}}\PY{n}{x\PYZus{}0}\PY{p}{)}\PY{o}{/}\PY{n}{dx}\PY{p}{)}\PY{p}{)}\PY{o}{+}\PY{l+m+mi}{1}\PY{p}{,}\PY{p}{(}\PY{n+nb}{int}\PY{p}{(}\PY{p}{(}\PY{n}{y\PYZus{}l}\PY{o}{\PYZhy{}}\PY{n}{y\PYZus{}0}\PY{p}{)}\PY{o}{/}\PY{n}{dy}\PY{p}{)}\PY{p}{)}\PY{o}{+}\PY{l+m+mi}{1}\PY{p}{)}\PY{p}{)}

    \PY{c+c1}{\PYZsh{} Applying the boundary conditions to u}
    \PY{k}{for} \PY{n}{i} \PY{o+ow}{in} \PY{n+nb}{range}\PY{p}{(}\PY{n}{u}\PY{o}{.}\PY{n}{shape}\PY{p}{[}\PY{l+m+mi}{0}\PY{p}{]}\PY{p}{)}\PY{p}{:}
        \PY{k}{for} \PY{n}{j} \PY{o+ow}{in} \PY{n+nb}{range}\PY{p}{(}\PY{n}{u}\PY{o}{.}\PY{n}{shape}\PY{p}{[}\PY{l+m+mi}{1}\PY{p}{]}\PY{p}{)}\PY{p}{:}
            \PY{n}{u}\PY{p}{[}\PY{n}{i}\PY{p}{,}\PY{n}{j}\PY{p}{]} \PY{o}{=} \PY{n}{g}\PY{p}{(}\PY{n}{x}\PY{p}{[}\PY{n}{i}\PY{p}{]}\PY{p}{,}\PY{n}{y}\PY{p}{[}\PY{n}{j}\PY{p}{]}\PY{p}{)}

    \PY{c+c1}{\PYZsh{} Solving the system of equation using SD Method}
    \PY{c+c1}{\PYZsh{} NOTE: b will be modified after the excution of this function}
    \PY{n}{SD\PYZus{}Solver}\PY{p}{(}\PY{n}{b}\PY{p}{,}\PY{n}{u\PYZus{}linear}\PY{p}{,}\PY{n}{U}\PY{p}{,}\PY{n}{Error}\PY{p}{,}\PY{n}{Tolerance}\PY{p}{,}\PY{n}{f}\PY{p}{,}\PY{n}{x}\PY{p}{,}\PY{n}{y}\PY{p}{,}\PY{n}{h}\PY{p}{,}\PY{n}{n}\PY{p}{,}\PY{n}{n\PYZus{}x}\PY{p}{,}\PY{n}{n\PYZus{}y}\PY{p}{)}

    \PY{c+c1}{\PYZsh{} Number of iterations required}
    \PY{n}{iteraions} \PY{o}{=} \PY{n+nb}{len}\PY{p}{(}\PY{n}{Error}\PY{p}{[}\PY{n}{h}\PY{p}{]}\PY{p}{)}

    \PY{c+c1}{\PYZsh{} Printing the value of h}
    \PY{n+nb}{print}\PY{p}{(}\PY{l+s+sa}{f}\PY{l+s+s2}{\PYZdq{}}\PY{l+s+s2}{For h = }\PY{l+s+si}{\PYZob{}}\PY{n}{h}\PY{l+s+si}{\PYZcb{}}\PY{l+s+s2}{:}\PY{l+s+s2}{\PYZdq{}}\PY{p}{)}
    
    \PY{c+c1}{\PYZsh{} Printing the number of iterations required}
    \PY{n+nb}{print}\PY{p}{(}\PY{l+s+sa}{f}\PY{l+s+s2}{\PYZdq{}}\PY{l+s+s2}{Number of iterations required: }\PY{l+s+si}{\PYZob{}}\PY{n}{iteraions}\PY{l+s+si}{\PYZcb{}}\PY{l+s+se}{\PYZbs{}n}\PY{l+s+se}{\PYZbs{}n}\PY{l+s+s2}{\PYZdq{}}\PY{p}{)}
\end{Verbatim}
\end{tcolorbox}

    \begin{Verbatim}[commandchars=\\\{\}]
For h = 0.1:
Number of iterations required: 297


For h = 0.05:
Number of iterations required: 1102


For h = 0.025:
Number of iterations required: 3986


    \end{Verbatim}

    \begin{tcolorbox}[breakable, size=fbox, boxrule=1pt, pad at break*=1mm,colback=cellbackground, colframe=cellborder]
\prompt{In}{incolor}{22}{\boxspacing}
\begin{Verbatim}[commandchars=\\\{\}]
\PY{k}{for} \PY{n}{h}\PY{p}{,}\PY{n}{error} \PY{o+ow}{in} \PY{n}{Error}\PY{o}{.}\PY{n}{items}\PY{p}{(}\PY{p}{)}\PY{p}{:}
    \PY{n}{plt}\PY{o}{.}\PY{n}{semilogy}\PY{p}{(}\PY{n}{error}\PY{p}{,}\PY{n}{label} \PY{o}{=} \PY{l+s+s2}{\PYZdq{}}\PY{l+s+s2}{h = }\PY{l+s+s2}{\PYZdq{}}\PY{o}{+}\PY{n+nb}{str}\PY{p}{(}\PY{n}{h}\PY{p}{)}\PY{p}{)}
\PY{n}{plt}\PY{o}{.}\PY{n}{legend}\PY{p}{(}\PY{p}{)}
\PY{c+c1}{\PYZsh{} NOTE: Here, the iteration index k starts from 0}
\PY{n}{plt}\PY{o}{.}\PY{n}{xlabel}\PY{p}{(}\PY{l+s+s2}{\PYZdq{}}\PY{l+s+s2}{Iteration index (k)}\PY{l+s+s2}{\PYZdq{}}\PY{p}{)}
\PY{n}{plt}\PY{o}{.}\PY{n}{ylabel}\PY{p}{(}\PY{l+s+s2}{\PYZdq{}}\PY{l+s+s2}{Relative Error}\PY{l+s+s2}{\PYZdq{}}\PY{p}{)}
\PY{n}{plt}\PY{o}{.}\PY{n}{title}\PY{p}{(}\PY{l+s+sa}{f}\PY{l+s+s2}{\PYZdq{}}\PY{l+s+s2}{Relative Error vs Iteration index (k)}\PY{l+s+s2}{\PYZdq{}}\PY{p}{)}
\PY{n}{plt}\PY{o}{.}\PY{n}{grid}\PY{p}{(}\PY{p}{)}
\PY{n}{plt}\PY{o}{.}\PY{n}{show}\PY{p}{(}\PY{p}{)}
\end{Verbatim}
\end{tcolorbox}

    \begin{center}
    \adjustimage{max size={0.9\linewidth}{0.9\paperheight}}{output_85_0.png}
    \end{center}
    { \hspace*{\fill} \\}
    
    \begin{center}\rule{0.5\linewidth}{0.5pt}\end{center}

    \hypertarget{for-each-method-sd-and-cg-for-each-h-show-the-3d-surface-plot-of-the-final-solution-u-as-a-function-of-x-and-y.}{%
\subsection{2. For each method (SD and CG), for each ''h'', show the 3D
surface plot of the final solution u (as a function of x and
y).}\label{for-each-method-sd-and-cg-for-each-h-show-the-3d-surface-plot-of-the-final-solution-u-as-a-function-of-x-and-y.}}

    \begin{center}\rule{0.5\linewidth}{0.5pt}\end{center}

    \hypertarget{answer-2-steepest-descent-sd}{%
\subsection{Answer (2): Steepest Descent
(SD)}\label{answer-2-steepest-descent-sd}}

    \begin{tcolorbox}[breakable, size=fbox, boxrule=1pt, pad at break*=1mm,colback=cellbackground, colframe=cellborder]
\prompt{In}{incolor}{23}{\boxspacing}
\begin{Verbatim}[commandchars=\\\{\}]
\PY{k}{for} \PY{n}{h}\PY{p}{,}\PY{n}{u} \PY{o+ow}{in} \PY{n}{U}\PY{o}{.}\PY{n}{items}\PY{p}{(}\PY{p}{)}\PY{p}{:}
    \PY{n}{fig} \PY{o}{=} \PY{n}{plt}\PY{o}{.}\PY{n}{figure}\PY{p}{(}\PY{p}{)}
    \PY{n}{ax} \PY{o}{=} \PY{n}{fig}\PY{o}{.}\PY{n}{add\PYZus{}subplot}\PY{p}{(}\PY{n}{projection} \PY{o}{=} \PY{l+s+s2}{\PYZdq{}}\PY{l+s+s2}{3d}\PY{l+s+s2}{\PYZdq{}}\PY{p}{)}
    \PY{n}{surf} \PY{o}{=} \PY{n}{ax}\PY{o}{.}\PY{n}{plot\PYZus{}surface}\PY{p}{(}\PY{n}{X\PYZus{}Y\PYZus{}dict}\PY{p}{[}\PY{n}{h}\PY{p}{]}\PY{p}{[}\PY{l+m+mi}{0}\PY{p}{]}\PY{p}{,} \PY{n}{X\PYZus{}Y\PYZus{}dict}\PY{p}{[}\PY{n}{h}\PY{p}{]}\PY{p}{[}\PY{l+m+mi}{1}\PY{p}{]}\PY{p}{,} \PY{n}{u}\PY{p}{,} \PY{n}{cmap}\PY{o}{=}\PY{n}{cm}\PY{o}{.}\PY{n}{summer}\PY{p}{,}\PY{n}{linewidth}\PY{o}{=}\PY{l+m+mi}{0}\PY{p}{)}
    \PY{n}{fig}\PY{o}{.}\PY{n}{colorbar}\PY{p}{(}\PY{n}{surf}\PY{p}{,} \PY{n}{shrink}\PY{o}{=}\PY{l+m+mf}{0.5}\PY{p}{,} \PY{n}{aspect}\PY{o}{=}\PY{l+m+mi}{5}\PY{p}{)}
    \PY{n}{ax}\PY{o}{.}\PY{n}{set\PYZus{}xlabel}\PY{p}{(}\PY{l+s+s2}{\PYZdq{}}\PY{l+s+s2}{x}\PY{l+s+s2}{\PYZdq{}}\PY{p}{)}
    \PY{n}{ax}\PY{o}{.}\PY{n}{set\PYZus{}ylabel}\PY{p}{(}\PY{l+s+s2}{\PYZdq{}}\PY{l+s+s2}{y}\PY{l+s+s2}{\PYZdq{}}\PY{p}{)}
    \PY{n}{ax}\PY{o}{.}\PY{n}{set\PYZus{}zlabel}\PY{p}{(}\PY{l+s+s2}{\PYZdq{}}\PY{l+s+s2}{u}\PY{l+s+s2}{\PYZdq{}}\PY{p}{)}
    \PY{n}{ax}\PY{o}{.}\PY{n}{set\PYZus{}title}\PY{p}{(}\PY{l+s+sa}{f}\PY{l+s+s2}{\PYZdq{}}\PY{l+s+s2}{Steepest Descent (SD): h = }\PY{l+s+si}{\PYZob{}}\PY{n}{h}\PY{l+s+si}{\PYZcb{}}\PY{l+s+s2}{\PYZdq{}}\PY{p}{)}
    \PY{n}{ax}\PY{o}{.}\PY{n}{set\PYZus{}box\PYZus{}aspect}\PY{p}{(}\PY{n}{aspect}\PY{o}{=}\PY{k+kc}{None}\PY{p}{,} \PY{n}{zoom}\PY{o}{=}\PY{l+m+mf}{0.75}\PY{p}{)}
    \PY{n}{ax}\PY{o}{.}\PY{n}{patch}\PY{o}{.}\PY{n}{set\PYZus{}edgecolor}\PY{p}{(}\PY{l+s+s1}{\PYZsq{}}\PY{l+s+s1}{black}\PY{l+s+s1}{\PYZsq{}}\PY{p}{)}  
    \PY{n}{ax}\PY{o}{.}\PY{n}{patch}\PY{o}{.}\PY{n}{set\PYZus{}linewidth}\PY{p}{(}\PY{l+m+mi}{1}\PY{p}{)}  
\PY{n}{plt}\PY{o}{.}\PY{n}{show}\PY{p}{(}\PY{p}{)}
\end{Verbatim}
\end{tcolorbox}

    \begin{center}
    \adjustimage{max size={0.9\linewidth}{0.9\paperheight}}{output_90_0.png}
    \end{center}
    { \hspace*{\fill} \\}
    
    \begin{center}
    \adjustimage{max size={0.9\linewidth}{0.9\paperheight}}{output_90_1.png}
    \end{center}
    { \hspace*{\fill} \\}
    
    \begin{center}
    \adjustimage{max size={0.9\linewidth}{0.9\paperheight}}{output_90_2.png}
    \end{center}
    { \hspace*{\fill} \\}
    
    \begin{center}\rule{0.5\linewidth}{0.5pt}\end{center}

    \hypertarget{a-table-comparing-the-iterations-required-to-reach-the-convergence-criteria-for-jacobi-gauss-seidel-steepest-descent-and-conjugate-gradient-for-all-three-h-values.}{%
\subsection{3. A table comparing the iterations required to reach the
convergence criteria for Jacobi, Gauss-Seidel, Steepest Descent and
Conjugate gradient for all three ''h''
values.}\label{a-table-comparing-the-iterations-required-to-reach-the-convergence-criteria-for-jacobi-gauss-seidel-steepest-descent-and-conjugate-gradient-for-all-three-h-values.}}

    \begin{center}\rule{0.5\linewidth}{0.5pt}\end{center}

    \hypertarget{answer-3}{%
\subsection{Answer (3):}\label{answer-3}}

    \hypertarget{iterations-required-to-reach-the-convergence-criteria-for-jacobi-gauss-seidel-steepest-descent-and-conjugate-gradient}{%
\subsection{Iterations required to reach the convergence criteria for
Jacobi, Gauss-Seidel, Steepest Descent and Conjugate
gradient}\label{iterations-required-to-reach-the-convergence-criteria-for-jacobi-gauss-seidel-steepest-descent-and-conjugate-gradient}}

    \[
\begin{aligned}
& \text {}\\
&\begin{array}{|c|c|c|c|c|}
\hline \hline \text { h } & \text { Jacobi } & \text { Gauss-Seidel } & \text { Steepest Descent }  & \text { Conjugate gradient } \\
\hline \frac{1}{10} & 293 & 157 & 297 & 29\\
\hline
\frac{1}{20} & 1077 & 574 & 1102 & 60\\
\hline
\frac{1}{40} & 3882 & 2069 & 3986 & 118\\
\hline
\end{array}
\end{aligned}
\]

    \begin{tcolorbox}[breakable, size=fbox, boxrule=1pt, pad at break*=1mm,colback=cellbackground, colframe=cellborder]
\prompt{In}{incolor}{24}{\boxspacing}
\begin{Verbatim}[commandchars=\\\{\}]
\PY{n}{J} \PY{o}{=} \PY{p}{[}\PY{l+m+mi}{293}\PY{p}{,}\PY{l+m+mi}{1077}\PY{p}{,}\PY{l+m+mi}{3882}\PY{p}{]}
\PY{n}{GS} \PY{o}{=} \PY{p}{[}\PY{l+m+mi}{157}\PY{p}{,}\PY{l+m+mi}{574}\PY{p}{,}\PY{l+m+mi}{2069}\PY{p}{]}
\PY{n}{SD} \PY{o}{=} \PY{p}{[}\PY{l+m+mi}{297}\PY{p}{,}\PY{l+m+mi}{1102}\PY{p}{,}\PY{l+m+mi}{3986}\PY{p}{]}
\PY{n}{CG} \PY{o}{=} \PY{p}{[}\PY{l+m+mi}{29}\PY{p}{,}\PY{l+m+mi}{60}\PY{p}{,}\PY{l+m+mi}{118}\PY{p}{]}
\end{Verbatim}
\end{tcolorbox}

    \begin{tcolorbox}[breakable, size=fbox, boxrule=1pt, pad at break*=1mm,colback=cellbackground, colframe=cellborder]
\prompt{In}{incolor}{25}{\boxspacing}
\begin{Verbatim}[commandchars=\\\{\}]
\PY{n}{plt}\PY{o}{.}\PY{n}{loglog}\PY{p}{(}\PY{n}{H}\PY{p}{,}\PY{n}{J}\PY{p}{,}\PY{l+s+s2}{\PYZdq{}}\PY{l+s+s2}{o\PYZhy{}}\PY{l+s+s2}{\PYZdq{}}\PY{p}{,}\PY{n}{label} \PY{o}{=} \PY{l+s+s2}{\PYZdq{}}\PY{l+s+s2}{Jacobi}\PY{l+s+s2}{\PYZdq{}}\PY{p}{)}
\PY{n}{plt}\PY{o}{.}\PY{n}{loglog}\PY{p}{(}\PY{n}{H}\PY{p}{,}\PY{n}{GS}\PY{p}{,}\PY{l+s+s2}{\PYZdq{}}\PY{l+s+s2}{d\PYZhy{}}\PY{l+s+s2}{\PYZdq{}}\PY{p}{,}\PY{n}{label} \PY{o}{=} \PY{l+s+s2}{\PYZdq{}}\PY{l+s+s2}{Gauss\PYZhy{}Seidel}\PY{l+s+s2}{\PYZdq{}}\PY{p}{)}
\PY{n}{plt}\PY{o}{.}\PY{n}{loglog}\PY{p}{(}\PY{n}{H}\PY{p}{,}\PY{n}{SD}\PY{p}{,}\PY{l+s+s2}{\PYZdq{}}\PY{l+s+s2}{\PYZgt{}\PYZhy{}}\PY{l+s+s2}{\PYZdq{}}\PY{p}{,}\PY{n}{label} \PY{o}{=} \PY{l+s+s2}{\PYZdq{}}\PY{l+s+s2}{Steepest Descent}\PY{l+s+s2}{\PYZdq{}}\PY{p}{)}
\PY{n}{plt}\PY{o}{.}\PY{n}{loglog}\PY{p}{(}\PY{n}{H}\PY{p}{,}\PY{n}{CG}\PY{p}{,}\PY{l+s+s2}{\PYZdq{}}\PY{l+s+s2}{\PYZca{}\PYZhy{}}\PY{l+s+s2}{\PYZdq{}}\PY{p}{,}\PY{n}{label} \PY{o}{=} \PY{l+s+s2}{\PYZdq{}}\PY{l+s+s2}{Conjugate Gradient}\PY{l+s+s2}{\PYZdq{}}\PY{p}{)}
\PY{n}{plt}\PY{o}{.}\PY{n}{legend}\PY{p}{(}\PY{p}{)}
\PY{n}{plt}\PY{o}{.}\PY{n}{grid}\PY{p}{(}\PY{n}{which}\PY{o}{=}\PY{l+s+s1}{\PYZsq{}}\PY{l+s+s1}{minor}\PY{l+s+s1}{\PYZsq{}}\PY{p}{)}
\PY{n}{plt}\PY{o}{.}\PY{n}{xlabel}\PY{p}{(}\PY{l+s+s2}{\PYZdq{}}\PY{l+s+s2}{h}\PY{l+s+s2}{\PYZdq{}}\PY{p}{)}
\PY{n}{plt}\PY{o}{.}\PY{n}{ylabel}\PY{p}{(}\PY{l+s+s2}{\PYZdq{}}\PY{l+s+s2}{Iterations}\PY{l+s+s2}{\PYZdq{}}\PY{p}{)}
\PY{n}{plt}\PY{o}{.}\PY{n}{show}\PY{p}{(}\PY{p}{)}
\end{Verbatim}
\end{tcolorbox}

    \begin{center}
    \adjustimage{max size={0.9\linewidth}{0.9\paperheight}}{output_98_0.png}
    \end{center}
    { \hspace*{\fill} \\}
    
    \hypertarget{observations}{%
\subsection{Observations:}\label{observations}}

    \hypertarget{conjugate-gradient-method-requires-considerably-less-iterations-as-compared-to-other-methods}{%
\subsubsection{1. Conjugate Gradient method requires considerably less
iterations as compared to other
methods}\label{conjugate-gradient-method-requires-considerably-less-iterations-as-compared-to-other-methods}}

    \hypertarget{steepest-descent-method-is-the-slowest-method-among-all-the-reported-methods.}{%
\subsubsection{2. Steepest Descent method is the slowest method among
all the reported
methods.}\label{steepest-descent-method-is-the-slowest-method-among-all-the-reported-methods.}}

    \hypertarget{jacobi-method-is-marginally-faster-as-compared-to-the-steepest-descent-method.}{%
\subsubsection{3. Jacobi method is marginally faster as compared to the
Steepest Descent
method.}\label{jacobi-method-is-marginally-faster-as-compared-to-the-steepest-descent-method.}}

    \hypertarget{gauss-siedel-method-is-approximately-twice-as-fast-as-the-jacobi-and-the-steepest-descent-methods.}{%
\subsubsection{4. Gauss-Siedel method is approximately twice as fast as
the Jacobi and the Steepest Descent
methods.}\label{gauss-siedel-method-is-approximately-twice-as-fast-as-the-jacobi-and-the-steepest-descent-methods.}}

    \hypertarget{as-h-reduces-the-number-of-iterations-required-for-conjugate-gradient-method-increased-at-a-slower-rate-as-compared-to-the-other-methods}{%
\subsubsection{5. As h reduces, the number of iterations required for
Conjugate Gradient method increased at a slower rate as compared to the
other
methods}\label{as-h-reduces-the-number-of-iterations-required-for-conjugate-gradient-method-increased-at-a-slower-rate-as-compared-to-the-other-methods}}

    \hypertarget{the-rate-of-increase-in-the-iterations-required-for-jacobi-gauss-seidel-and-steepest-descent-is-almost-the-same.-the-slopes-are-almost-equal-for-these-methods}{%
\subsubsection{6. The rate of increase in the iterations required for
Jacobi, Gauss-Seidel and Steepest Descent is almost the same. (The
slopes are almost equal for these
methods)}\label{the-rate-of-increase-in-the-iterations-required-for-jacobi-gauss-seidel-and-steepest-descent-is-almost-the-same.-the-slopes-are-almost-equal-for-these-methods}}


    % Add a bibliography block to the postdoc
    
    
    
\end{document}
